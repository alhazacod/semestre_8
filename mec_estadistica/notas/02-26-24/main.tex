\documentclass{article}

\usepackage[most]{tcolorbox}
\usepackage{physics}
\usepackage{graphicx}
\usepackage{float}
\usepackage{amsmath}
\usepackage{amssymb}


\usepackage[utf8]{inputenc}
\usepackage[a4paper, margin=1in]{geometry} % Controla los márgenes
\usepackage{titling}

\title{Clase 6 }
\author{Manuel Garcia.}
\date{\today}

\renewcommand{\maketitlehooka}{%
  \centering
  \vspace*{0.05cm} % Espacio vertical antes del título
}

\renewcommand{\maketitlehookd}{%
  \vspace*{2cm} % Espacio vertical después de la fecha
}

\newcommand{\caja}[3]{%
  \begin{tcolorbox}[colback=#1!5!white,colframe=#1!25!black,title=#2]
    #3
  \end{tcolorbox}%
}

\begin{document}
\maketitle

\section{Ensable microcanínico Cuántico }
Ecuacion de valores propios 
\begin{gather*}
  H ^ {(N) } \phi_j (\vec q ) = E_j \phi_j (\vec q )
\end{gather*}
Se conocen las funciones de onda de probabilidad $ \phi_j (\vec q ) $ que describen los estados cuánticos $ \ket{\phi_j } $. Por lo tanto se conocen los estados cuánticos del sistema de $ N  $ particulas $ \{\ket{\phi_j }\} $ que es la base del espacio de hilbert $ \mathcal H ^ {(N) } $ y el espectro de niveles $ \{E_j\} $.

El interes es encontrar la probabilidad $ p_j  $ de que el sistema aislado se encuentre en un estado cuantico $ \ket{\phi_j } $ con energia $ E_j  $. Para el sistema aislado $ E= U _{sis }  $ es fija.



\subsection{Distribución de Probabilidad Microcanónica }
La probabilidad de que el sistema aislado se encuentre en un microestado $ \ket{\phi_j } $ con ejercia $ E_j  $ 
\begin{gather*}
  p_j = p_j(E_j) 
\end{gather*}
Como $ E  $ (energia del sistema) es fija entonces los estados accesibles del sistema 
\begin{gather*}
  E_l = E_s = E_m = \cdots = E_p = E  
\end{gather*}
Se cumple que todos los estados accesibles tienen la misma probabilidad 
\begin{gather*}
  p_j = p_s = p_m = \cdots = p_p  
\end{gather*}
Se cumple que todos los $ g=W  $ estados accesibles tienen igual probabilidad.
\begin{gather*}
  \displaystyle\sum_{j=1 }^{W }p_j = Wa = 1 \qquad \rightarrow \qquad a = \frac{1}{W } 
\end{gather*}
donde $ a  $ es la probabilidad del estado.
\begin{gather*}
  p_j =\left\{
    \begin{array}{lr}
      \frac{1}{W}, &\text{para } E_j = E \\
      0, &\text{para } E_j \neq E
    \end{array}
\right. 
\end{gather*}

\textbf{Operador matriz densidad de probabilidad del ensamble microcanínico } 
\begin{gather*}
  p_j = p(E_j) = \frac{1}{W } 
\end{gather*}
Teniendo en cuenta que de manera general la probabilidad de que en un instante dado un sistema se encuentre en el microestado $ \ket{\phi_j } $ con energia $ E_j  $ 
\begin{gather*}
  p_j = p(E_j ) = \frac{\left|b_j \right|^2 }{\displaystyle\sum_{j }^{} \left|b_j \right|^2 } 
\end{gather*}
entonces para el ensamble microcanonico 
\begin{gather*}
  \left|b_j \right|^2 = 1  
\end{gather*}
Entonces el operador matriz densidad de probabilidad $ \hat \rho $ 
\begin{gather*}
  p _{lj }  = \delta _{lj } \left|b_j \right|^2\\
  \text{Donde }\\
  \left|b_j \right|^2 = 
  \left\{\begin{array}{lr}
      1, &\text{para } E_j = E \\
      0, &\text{para } E_j \neq E
    \end{array}\right.
\end{gather*}
El operador $ \hat \rho $ se puede escribir 
\begin{gather*}
  \hat \rho = \displaystyle\sum_{j=1 }^{W } \ket{\phi_j }\left|b_j \right|^2 \bra{\phi_j } = \displaystyle\sum_{j=1 }^{W } \ket{\phi_j }\bra{\phi_j } 
\end{gather*}
Y tendremos que 
\begin{gather*}
  \Omega = Tr[\hat \rho] = W 
\end{gather*}
Donde $ \Omega  $ es la \textbf{funcion de partición} microcanónica.

\textbf{Promedio estadistico de la energia } (definicion estadistica de la energía interna)
\begin{gather*}
  \expval{E } = \frac{\displaystyle\sum_{j=1 }^{W }E_j p_j }{\displaystyle\sum_{j=1 }^{W } p_j } = \frac{Tr[\hat \rho \hat H ^ {(N )}]}{Tr[\hat \rho]} = E 
\end{gather*}
Y la fluctuacion estadistica de la energia 
\begin{gather*}
  \Delta E = \sqrt{\expval{E^2 } - \expval{E }^2 } = 0 
\end{gather*}
Esto es consecuencia de que la energía del sistema $ E  $ es fija.

\section{Definicion estadística de la entropía }
La conexion entre el ensamble microcanónico y la termodinámica se establece de manera general al definir estadísticamente a la entropía $ S(E,V,N ) $ mediante el principio de Boltzmann 
\begin{gather*}
  S(E,V,N ) = \kappa_B \ ln[W(E,V,N )] = \kappa_B \ ln[\Omega (E,V,N )]
\end{gather*}

\subsection{Incremento de la entropia en un gas ideal en expansion }
Los niveles de energia d euna particula libre en un pozo de potencial tridimensinal 
\begin{gather*}
  \epsilon _{i_x,i_y,i_z } = \epsilon _{i_x }  + \epsilon _{i_y } + \epsilon _{i_z }  = \frac{h^2 }{8m } \frac{1}{L^2 }(i_x^2+i_y^2+i_z^2) 
\end{gather*}
La energia 
\begin{gather*}
  E_j = \frac{h^2 }{8m } \frac{1}{L^2 } \displaystyle\sum_{l=1 }^{N }(i _{xl } ^2 +i _{yl } ^2+i _{zl } ^2) 
\end{gather*}
energia promedio 
\begin{gather*}
  \expval{\epsilon} = \frac{\expval{E }}{N } =\frac{h^2 }{4m^2 } \frac{1}{NL^2 } \expval{\displaystyle\sum_{l=1 }^{N }(i _{xl } ^2 +i _{yl } ^2+i _{zl } ^2)}
\end{gather*}
Lo que implica que 
\begin{gather*}
  \expval{v^2 } = \frac{\expval{p^2 }}{m^2 } = \frac{h }{2mL } \frac{1}{\sqrt{N } } \sqrt{\expval{\displaystyle\sum_{l=1 }^{N }(i _{xl } ^2 +i _{yl } ^2+i _{zl } ^2)} }
\end{gather*}
Dado que $ \expval{v } \leq c  $ entonces 
\begin{gather*}
   \sqrt{\expval{\displaystyle\sum_{l=1 }^{N }(i _{xl } ^2 +i _{yl } ^2+i _{zl } ^2)} } \leq \frac{2mcL \sqrt{N }}{h  }
\end{gather*}
Si tenemos $ V_2 > V_1  $ debido a que $ L_2 = L_1 + \Delta L  $ se cumple que 
\begin{gather*}
  W_2 >W_1  
\end{gather*}
Como el numero de estados aumenta la entropia aumenta para que la desigualdad se cumpla. Esto es consistente con la entropia de Boltzmann.

\begin{gather*}
  S_1 = \kappa_B\  log[W_1] \qquad \qquad \qquad S_2   = \kappa_B\  log[W_2]\\
  \hfill \\
  S_1 < S_2
\end{gather*}



\end{document}
