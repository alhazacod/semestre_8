\documentclass{article}

\usepackage[most]{tcolorbox}
\usepackage{physics}
\usepackage{graphicx}
\usepackage{float}
\usepackage{amsmath}
\usepackage{amssymb}


\usepackage[utf8]{inputenc}
\usepackage[a4paper, margin=1in]{geometry} % Controla los márgenes
\usepackage{titling}

\title{Clase Mecánica Estadística }
\author{Manuel Garcia.}
\date{\today}

\renewcommand{\maketitlehooka}{%
  \centering
  \vspace*{0.05cm} % Espacio vertical antes del título
}

\renewcommand{\maketitlehookd}{%
  \vspace*{2cm} % Espacio vertical después de la fecha
}

\newcommand{\caja}[3]{%
  \begin{tcolorbox}[colback=#1!5!white,colframe=#1!25!black,title=#2]
    #3
  \end{tcolorbox}%
}

\begin{document}
\maketitle

\section{Observables termodinámicos en el ensamble gran-canónico}
Se obtuvo que la probabilidad de que el sistema termodinámico de interés, en un instante dado de tiempo, se encuentre teniendo $ N  $ partículas en el estado $ \ket{\Phi_j } $ con energía $ E_j  $ es 
\begin{gather*}
  p _{Nj } = \frac{e ^ {-\beta E_j - \gamma N }}{Z_N } 
\end{gather*}
Donde la función de gran partición 
\begin{gather*}
  Z_N = Z(T,V,N) = \displaystyle\sum_{N }^{}\displaystyle\sum_{j }^{} e ^ {-\beta E_j - \gamma N } = \displaystyle\sum_{N }^{} \lambda^N Q_N  
\end{gather*}
Con fugacidad $ \lambda  $
\begin{gather*}
  \lambda = e ^ {-\gamma } 
\end{gather*}
Y el multiplicador de Lagrange 
\begin{gather*}
  \gamma = - \beta\mu 
\end{gather*}
con $ \beta = \frac{1}{\kappa_B T } $ y $ Q_N  $ la función de partición canónica del sistema de $ N  $ partículas dada por 
\begin{gather*}
  Q_N = Q(T,V,N) = \displaystyle\sum_{j }^{} e ^ {- \beta E_j } 
\end{gather*}

\hfill 

\hfill 

Los observables termodinámicos energía interna $ E  $ y numero de matriculas $ N  $
\begin{gather*}
   \expval{E } = \frac{\sum_{N }^{} \sum_{j }^{} E_j e ^ {-\beta E_j - \gamma N }}{\sum_{N }^{} \sum_{j }^{} e ^ {-\beta E_j - \gamma N }} = - \frac{\partial  }{\partial \beta } \left[\ln \displaystyle\sum_{N }^{} \displaystyle\sum_{j }^{} e ^ {-\beta E_j - \gamma N }\right] 
   \\
   \expval{N } = \frac{\sum_{N }^{} \sum_{j }^{} N e ^ {-\beta E_j - \gamma N }}{\sum_{N }^{} \sum_{j }^{} e ^ {-\beta E_j - \gamma N }} = - \frac{\partial  }{\partial \gamma } \left[\ln \displaystyle\sum_{N }^{} \displaystyle\sum_{j }^{} e ^ {-\beta E_j - \gamma N }\right]
\end{gather*}
En general, un observable termodinámico cualquiera $ \mathcal O  $ queda definido como el promedio estadístico del observable $ \expval{\mathcal O } $ en el ensamble gran-canónico, es decir 
\begin{gather*}
  \expval{\mathcal O } 
  = \frac{1}{Z_N } \displaystyle\sum_{N }^{} \displaystyle\sum_{j }^{} \mathcal O \lambda^N e ^ {- \beta E_j } 
  = \frac{1}{Z_N } \displaystyle\sum_{N }^{} \lambda^N \expval{\mathcal O }_N Q_N  
\end{gather*}
Con 
\begin{gather*}
  \expval{\mathcal O } 
  = \frac{Q_N }{2} \displaystyle\sum_{j }^{} \mathcal O e ^ {- \beta E_j } 
\end{gather*}






\section{Operador matriz densidad de probabilidad del ensamble gran-canónico }
El operador matriz densidad de probabilidad del ensamble gran-canónico está definido a partir de los elementos de matriz 
\begin{gather*}
  \rho _{ij }  = \delta _{lj }  e ^ {-\beta E_j + \beta_\mu N } 
\end{gather*}
De forma equivalente el operador matriz densidad de probabilidad queda definido como 
\begin{gather*}
  \hat \rho = \displaystyle\sum_{N }^{} \displaystyle\sum_{j }^{} \ket{\Phi_j^N } e ^ {- \beta E_j + \beta N_\mu } \bra{\Phi_j ^N } = e ^ {-\beta\hat H + \beta_\mu \hat N } \displaystyle\sum_{N }^{} \displaystyle\sum_{j }^{} \ket{\Phi_j^N } \bra{\Phi_j^N} 
\end{gather*}
Tal que se satisface que 
\begin{gather*}
   \displaystyle\sum_{N }^{} \displaystyle\sum_{k }^{} \ket{\Phi_k^N } \bra{\Phi_k^N} = 1
\end{gather*}
Por cual el \textbf{operador matriz densidad e probabilidad del ensamble gran-canónico } es  
\begin{gather*}
  \hat \rho = e ^ {-\beta\hat H + \beta_\mu \hat N } 
\end{gather*}
De tal forma que en el sistema hay $ m  $ especies de partículas 
\begin{gather*}
  \mu\hat N = \displaystyle\sum_{l= 1 }^{m } \mu_l \hat n _l  
\end{gather*}
En general se cumple que 
\begin{gather*}
  \left[ \hat{N} , \hat H   \right] = 0 \qquad \qquad \left[ \hat N_l  , \hat H   \right] = 0 \qquad \qquad \left[ \hat N_l  , \hat N_k   \right] = 0  
\end{gather*}

\hfill 

\hfill 

La función de gran partición se puede escribir en termino de $ \hat \rho  $ 
\begin{gather*}
  Z(T,V,N) = \Tr \hat \rho  = \displaystyle\sum_{N }^{} \displaystyle\sum_{j }^{} e ^ {- \beta E_j + \beta_\mu N  } 
\end{gather*}
así, el promedio estadístico de un observable $ \mathcal O  $ 
\begin{gather*}
  \expval{\mathcal O } = \frac{\Tr \left[\hat{\rho} \hat{\mathcal{O}} \right]}{\Tr \left[\hat \rho \right]}
\end{gather*}

\hfill 

\hfill 

\begin{gather*}
  N_l = \kappa_B T \frac{\partial \ln Z_N  }{\partial \mu_l } = \kappa_B T \frac{\partial \ln Z_{N_l } }{\partial \mu_l } 
\end{gather*} 





\section{}


\end{document}
