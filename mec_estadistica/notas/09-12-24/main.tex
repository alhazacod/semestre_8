\documentclass{article}

\usepackage[most]{tcolorbox}
\usepackage{physics}
\usepackage{graphicx}
\usepackage{float}
\usepackage{amsmath}
\usepackage{amssymb}
\usepackage{tikz-3dplot}


\usepackage[utf8]{inputenc}
\usepackage[a4paper, margin=1in]{geometry} % Controla los márgenes
\usepackage{titling}

\title{Distribución de Bose-Einstein }
\author{Manuel Garcia.}
\date{\today}

\renewcommand{\maketitlehooka}{%
  \centering
  \vspace*{0.05cm} % Espacio vertical antes del título
}

\renewcommand{\maketitlehookd}{%
  \vspace*{2cm} % Espacio vertical después de la fecha
}

\newcommand{\caja}[3]{%
  \begin{tcolorbox}[colback=#1!5!white,colframe=#1!25!black,title=#2]
    #3
  \end{tcolorbox}%
}

\begin{document}
\maketitle

\section{}
Vamos a tener una cavidad conductora cubica de arista $ l  $ con una placa conductora en $ L  $.

\hfill 

% Set up the 3D coordinate frame
\tdplotsetmaincoords{70}{110}  % Angle of 3D view

\begin{tikzpicture}[tdplot_main_coords]

    % Dimensions of the cubic cavity
    \def\L{5}  % Length of the cube side
    \def\Lp{2} % Position of the conductive plane (L' < L)

    % Draw the cubic cavity
    \draw[thick] (0,0,0) -- (\L,0,0) -- (\L,\L,0) -- (0,\L,0) -- cycle; % Bottom face
    \draw[thick] (0,0,\L) -- (\L,0,\L) -- (\L,\L,\L) -- (0,\L,\L) -- cycle; % Top face
    \draw[thick] (0,0,0) -- (0,0,\L); % Vertical edges
    \draw[thick] (\L,0,0) -- (\L,0,\L);
    \draw[thick] (\L,\L,0) -- (\L,\L,\L);
    \draw[thick] (0,\L,0) -- (0,\L,\L);

    % Draw the conductive plane (at L' on the y-axis, i.e., xz-plane)
    \draw[fill=gray,opacity=0.5] (0,\Lp,0) -- (\L,\Lp,0) -- (\L,\Lp,\L) -- (0,\Lp,\L) -- cycle;

    % Add labels for dimensions and the conductive plane
    \node at (\L/2,-0.5,0) {Length $l$};
    \node at (-0.5,\L/2,\L/2) {Height $l$};
    \node at (\L+0.5,\Lp,\L/2) {Conductive plane at $y = L$};

\end{tikzpicture}

\hfill

Con condición de frontera de que el campo en las placas debe ser nulo 
\begin{gather*}
  E_{||}(0,t) = E_{||}(l,t) = E_{||} (l - L,t) = 0  
\end{gather*}
\begin{gather*}
  E(l,t) = E_{0x} \sin(\omega_x t + \phi_x) \sin (k_x l ) = 0  \\
  E(l,t) = E_{0x} \sin(\omega_x t + \phi_x) \sin (k_x l ) = 0  \\
  E(l,t) = E_{0x} \sin(\omega_x t + \phi_x) \sin (k_x l ) = 0  
\end{gather*}
Se encuentra que 
\begin{gather*}
  k_x = k_y = \frac{n\pi }{l } \qquad \qquad k_z = \frac{n \pi }{L } 
\end{gather*}
Para la región II 

\end{document}

