\documentclass{article}

\usepackage[most]{tcolorbox}
\usepackage{physics}
\usepackage{graphicx}
\usepackage{float}
\usepackage{amsmath}
\usepackage{amssymb}


\usepackage[utf8]{inputenc}
\usepackage[a4paper, margin=1in]{geometry} % Controla los márgenes
\usepackage{titling}

\title{Clase 10 }
\author{Manuel Garcia.}
\date{\today}

\renewcommand{\maketitlehooka}{%
  \centering
  \vspace*{0.05cm} % Espacio vertical antes del título
}

\renewcommand{\maketitlehookd}{%
  \vspace*{2cm} % Espacio vertical después de la fecha
}

\newcommand{\caja}[3]{%
  \begin{tcolorbox}[colback=#1!5!white,colframe=#1!25!black,title=#2]
    #3
  \end{tcolorbox}%
}

\begin{document}
\maketitle

\section{Ensamble Estadístico Canónico de $ N  $ Partículas y Ejemplos}
\subsection{Fluctuacion de la Energía en el Ensamble Canónico }
El promedio estadistico de la energia 
\begin{gather*}
  \expval{E }
\end{gather*}
Y el promedio cuadrado 
\begin{gather*}
  \expval{E^2 } 
\end{gather*}
en terminos de la función de particion $ Q_N = Q(T,V,N ) $
\begin{gather*}
  \expval{E } = \frac{\sum_{j }^{} E_j e ^ {- \beta E_j }}{\sum_{j }^{} e ^ {- \beta E_j }} = - \frac{1}{Q_N } \frac{\partial Q_N  }{\partial \beta} \\
  \expval{E^2 } = \frac{\sum_{j }^{} E_j^2 e ^ {- \beta E_j }}{\sum_{j }^{} e ^ {- \beta E_j }} = - \frac{1}{Q_N } \frac{\partial^2 Q_N  }{\partial \beta^2 }
\end{gather*}
Donde $ Q_N = \sum_{j }^{} e ^ {-\beta E_j } $.

La fluctuacion de la energia 
\begin{align*}
  (\expval{\Delta })^2 &= \expval{E^2 } - \expval{E }^2 \\
  &= \frac{1}{Q_N } \frac{\partial ^2 Q_N  }{\partial \beta^2 } - \frac{1}{Q_N^2 } \left(\frac{\partial  Q_N  }{\partial \beta }\right) \\
  &= \frac{\partial x }{\partial \beta} \left(\frac{\partial \Phi  }{\partial \beta}\right) 
\end{align*}

\textbf{Potencial Canonico }
\begin{gather*}
  \log{Q_N } = S ^ {(0) } - \beta E \equiv \Phi = - \frac{1}{\kappa_B T }A = - \beta A 
\end{gather*}
\begin{gather*}
  S ^ {(0)} = \sum_{j }^{} p_j \log{p_j} 
\end{gather*}

Luego de mucha puta algebra se llega a 
\caja{green}{}{
  \begin{gather*}
    (\Delta E )^2 = - \frac{\partial E  }{\partial \beta} = \tau^2 c  
  \end{gather*}
}
Donde $ \tau $ es la energia termica y $ c  $ es el calor especifico. El carlor especifico en terminos del calor especifico por particula es $ c = N \bar c  $, donde $ \bar c  $ es el calor promedio especifico por particula.
\begin{gather*}
  \Delta E = \tau \sqrt{c }  = \tau \sqrt{N\bar c }  
\end{gather*}
$ E  $ se puede escribir en terminos d ela energia promedio por particula $ \bar \epsilon  $ 
\begin{gather*}
  \frac{\Delta E }{\expval{E }} = \frac{\tau \sqrt{N } \sqrt{\bar c } }{N \bar \epsilon} = \tau \frac{\sqrt{\bar c } }{\bar \epsilon} \frac{1}{\sqrt{N } } \approx \frac{1}{\sqrt{N } } 
\end{gather*}

\subsection{Equivalencia entre los ensambles microcanonico y canonico }
Del principio de boltzman 
\begin{gather*}
  S = \kappa_B \ln{W } = \beta_B \ln{D(E)\Delta E }
\end{gather*}
Siendo $ D(E)  $ la densidad de estados y $ \Delta E  $ el rango de nergía contenido en el intervalor $ (E - \Delta E/2 , E + \Delta E/2 ) $. La función de partición canónica para el sistema macroscópico de N partículas también se puede definir como 
\begin{gather*}
  Q_N = \displaystyle\sum_{j }^{} e ^ {- E_j /\kappa_B T } = \displaystyle\int_{}^{} e ^ {- E/\kappa_B T }D(E) dE \\
  \expval{E } = \displaystyle\sum_{j }^{} E_j p_j = \frac{\displaystyle\int_{}^{} Ee ^ {- E/\kappa_B T }D(E) dE}{\displaystyle\int_{}^{} e ^ {- E/\kappa_B T }D(E) dE}
\end{gather*}
Como el denominador corresponde a la funcion de particion 
\begin{gather*}
  Q_N \approx e ^ {- U /\kappa_B T }D(U) \Delta E  
\end{gather*}
y 
\begin{gather*}
   \displaystyle\int_{}^{} Ee ^ {- E/\kappa_B T }D(E) dE \approx e ^ {- U /\kappa_B T }D(U) \Delta E
\end{gather*}


\end{document}
