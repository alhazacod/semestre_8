\documentclass{article}

\usepackage[most]{tcolorbox}
\usepackage{physics}
\usepackage{graphicx}
\usepackage{float}
\usepackage{amsmath}
\usepackage{amssymb}


\usepackage[utf8]{inputenc}
\usepackage[a4paper, margin=1in]{geometry} % Controla los márgenes
\usepackage{titling}

\title{Clase Mecanica Estadistica }
\author{Manuel Garcia.}
\date{\today}

\renewcommand{\maketitlehooka}{%
  \centering
  \vspace*{0.05cm} % Espacio vertical antes del título
}

\renewcommand{\maketitlehookd}{%
  \vspace*{2cm} % Espacio vertical después de la fecha
}

\newcommand{\caja}[3]{%
  \begin{tcolorbox}[colback=#1!5!white,colframe=#1!25!black,title=#2]
    #3
  \end{tcolorbox}%
}

\begin{document}
\maketitle

\section{Ensamble canónico para sistema económico }
Dos cantidades conservada: 
\begin{itemize}
  \item $ N  $ : numero de agenes económicos 
  
  \item $ M  $: Cantidad de dinero total
\end{itemize}
El dinero promedio por agencia $ \frac{N}{M}  $ tambien es una cantidad conservada.

Cada agente posee una cantidad de dinero ubicada en un rango $ m_1,m_2,\cdots,m_i  $.

La cantidad de agentes se ecnuentra en un ranto $ n_1,n_2,\cdots,n_i  $.

El número de configuraciones o arreglos estadísticos es $ W = \frac{N!}{n_!n_2!\cdots n_i! } $.

$ N = \bar m N = cte  $

$ T = \bar m  $ Dinero promedio por agente 

No existe analogia con el volumen $ V  $

$ M  $: dinero total se conserva 

$ N  $ Numero de agentes se conserva 

En estado de equilibrio el numero de configuraciones: $ W = W_{max } $
\begin{gather*}
  W_{max} = \frac{N! }{n_1!n_2! \cdots n_i!} 
\end{gather*}

\hfill 

\hfill 

Entropia de shannonn $ S^0  $ 
\begin{gather*}
  S^0 = \ln{W_{max }} = N \ln N - N \displaystyle\sum_{}^{} ...
\end{gather*}

Condiciones de ligadura 
\begin{gather}
  \displaystyle\int_{0 }^{\infty} p(m) dm = 1
\end{gather}


\end{document}
