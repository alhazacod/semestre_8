\documentclass{article}

\usepackage[most]{tcolorbox}
\usepackage{physics}
\usepackage{graphicx}
\usepackage{float}
\usepackage{amsmath}
\usepackage{amssymb}


\usepackage[utf8]{inputenc}
\usepackage[a4paper, margin=1in]{geometry} % Controla los márgenes
\usepackage{titling}

\title{Clase 5}
\author{Manuel Garcia.}
\date{\today}

\renewcommand{\maketitlehooka}{%
  \centering
  \vspace*{0.05cm} % Espacio vertical antes del título
}

\renewcommand{\maketitlehookd}{%
  \vspace*{2cm} % Espacio vertical después de la fecha
}

\newcommand{\caja}[3]{%
  \begin{tcolorbox}[colback=#1!5!white,colframe=#1!25!black,title=#2]
    #3
  \end{tcolorbox}%
}

\begin{document}
\maketitle

\section{Formulación Cuántica del Teorema de Liouville}
\begin{gather*}
  i \hbar \frac{\partial \tilde \rho }{\partial t } = [\hat H ^ {(N) }, \hat \rho] 
\end{gather*}
Cuando $ \hat \rho $ es una funcion de la energia 
\begin{gather*}
  \hat \rho = f(\hat H ^ {(N )}) \qquad \rightarrow \qquad \frac{\partial \hat \rho }{\partial t } = 0
\end{gather*}
dado que $ [\hat H ^ {(N) }, \hat \rho] = 0 $.

Por el contrario cuando $ \hat \rho  $ depende del tiempo 
\begin{gather*}
  i \hbar  \frac{\partial \hat \rho }{\partial t } = \left[  \hat H ^{(N)} , \hat \rho  \right] \neq 0 
\end{gather*}

\hfill 

\hfill 

\hfill 

\hfill 



En el espacio de fase $ 2f $ dimensional del sistema resulta posible definir una hipersuperficie $ \omega(E) $ de dimension $ 2f-1 $, mediante la siguiente integral 
\begin{gather*}
  \omega(E) = \displaystyle\int_{E = H(p_1,\cdots , p _{3n } ; q_1,\cdots, q _{2N } )}^{} d\omega 
\end{gather*}
Si $ \Delta \Lambda $ es un elemento de volumen del espacio de fase entonces un elemento d ehipersuperficie $ \Delta \omega $ está definido en un espacio de fase de dimension $ 2f-1  $. La hipersuperficie de fase $ \omega(E)  $ de dimensión $ 2f-1  $ envuelve un volumen $ \Lambda $ en el espacio de fase de dumensión $ 2f  $.
\begin{gather*}
  \Lambda(E) = \displaystyle\int_{H(p_1, \cdots, p_f;q_1, \cdots, q_f)\leq E }^{}d\Lambda 
\end{gather*}

\hfill 

\hfill 


\textbf{Numero de microestados} Tenemos $ E  $ variable y $ V,N  $ fijos. Existe un numero $ W(E,V,N) $ que se determina en términos de la hipersuperficie $ \omega(E,V,N ) $
\begin{gather*}
   W(E,V,N) = \frac{\omega(E,V,N)}{\omega_0}
\end{gather*}
Es de esperar que las propiedades termodinámicas del sistema no dependan de la constante d eproporcionalidad $ \omega_o ^ {-1 } $. Razon de cambio 
\begin{gather*}
  \frac{W_2 }{W_1 } =  \frac{W (E_2,V,N)}{W (E_1,V,N)} = \frac{\omega(E_2,V,N)}{\omega(E_2,V,N  )}
\end{gather*}
El volumen 
\begin{gather*}
  \Lambda(E,V,N) = \displaystyle\int_{H(p_1, \cdots, p_f;q_1, \cdots, q_f)\leq E }^{}dq_1\cdots dq_f dp_1 \cdots dp_f
\end{gather*}
Un cambio en la energia interna del sistema $ \Delta E  $ genera un cambio en el area de la hipersuperficie 
\begin{gather*}
  \Delta\omega = \omega(E_2,V,N) - \omega(E_1,V,N) 
\end{gather*}
Dando lugar a 
\begin{gather*}
  \Delta\Lambda = \Lambda(E + \Delta E, V,N ) - \Lambda (E,V,N ) = \left(\frac{\partial \Lambda }{\partial E }\right) _{V,N } \Delta E 
\end{gather*}
Por el teorema de Cavalleri 
\begin{gather*}
  \Delta\Lambda = \omega(E) \Delta E 
\end{gather*}
Entonces 
\begin{gather*}
  \omega(E) = \frac{\partial \Lambda(E)  }{\partial E}  
\end{gather*}
Por lo tanto 
\begin{gather*}
  W(E,V,N) = \frac{\omega(E,V,N )}{\omega_0 } = \frac{1}{\omega_0 } \frac{\partial \Lambda }{\partial E } 
\end{gather*}

\textbf{Volumen en el espacio de fase para el gas ideal de particulas libres }
\begin{gather*}
  V = \int_{}^{} d^2 q \qquad \rightarrow \qquad \int_{}^{}dq_1 \cdots dq _{3N }  = V ^ {N } 
\end{gather*}
Por lo tanto 
\begin{gather*}
  \Lambda (E,V,N)=V^N \int_{H(p,q) \leq E }^{} dp_1 \cdots dp _{3N } = V^N V ^ {\beta} _{3N }  
\end{gather*}
\begin{gather*}
  V ^ {\beta } _{3N }  = \int_{H(p,q) \leq E }^{} dp_1 \cdots dp _{3N }  
\end{gather*}


\end{document}
