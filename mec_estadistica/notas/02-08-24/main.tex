\documentclass{article}

\usepackage[most]{tcolorbox}
\usepackage{physics}
\usepackage{graphicx}
\usepackage{float}
\usepackage{amsmath}
\usepackage{amssymb}


\usepackage[utf8]{inputenc}
\usepackage[a4paper, margin=1in]{geometry} % Controla los márgenes
\usepackage{titling}

\title{Clase 2 }
\author{Manuel Garcia.}
\date{\today}

\renewcommand{\maketitlehooka}{%
  \centering
  \vspace*{0.05cm} % Espacio vertical antes del título
}

\renewcommand{\maketitlehookd}{%
  \vspace*{2cm} % Espacio vertical después de la fecha
}

\newcommand{\caja}[3]{%
  \begin{tcolorbox}[colback=#1!5!white,colframe=#1!25!black,title=#2]
    #3
  \end{tcolorbox}%
}

\begin{document}
\maketitle

\section{Mecanica Clasica}
Energia de gas ideal: 
\begin{gather*}
  U = \frac{1}{2}m_1v_1 ^2 + ... = \displaystyle\sum_{j=1 }^{N } \frac{1 }{2} m_j ^2 v_j ^2 
\end{gather*}
La energia interna del sistema se puede escirbir como $ U = N \bar \epsilon $, con la energia promedio por partícula $ \bar \epsilon  $ dada por 
\begin{gather*}
  \bar \epsilon = \frac{U }{N } = \frac{1}{2}m \vec v^2 
\end{gather*}
Definimos un parámetro fundamental denominado temperatura cinética del gas ideal $ T  $ de tal forma que la energia cinética promedio por partícula es proporcional a este parámetro, es decir 
\begin{gather*}
  \bar \epsilon = \frac{3 }{2} \kappa_B T  
\end{gather*}
Podemos escribir la energía interna en términos de la temperatura cinética, es decir 
\begin{gather*}
  U = N \bar \epsilon = \frac{3}{2} N \kappa_B T 
\end{gather*}

\hfill 

\hfill

La persion ejercida por un gas ideal en las paredes se origina en los impulsos ejercidos por las partículas que chocan contra las paredes. La presion sobre la pared es 
\begin{gather*}
  P = \frac{2}{3 } n @bar \epsilon  =  n \kappa_B T 
\end{gather*}
Teniendo en cuenta que la densidad columétrica de partículas $ n  $ estça definida como 
\begin{gather*}
  n = \frac{N }{V }  
\end{gather*}
\textbf{Ecuación de estado de los gases ideales }
\begin{gather*}
  PV = \kappa_B NT 
\end{gather*}

\textbf{Formulación analitica f grados de libertad } En la formulaicón lagrngiana, un sistema fisico conservativo con $ f  $ grados de libertad está descrito por 
\begin{gather*}
  L ^ {(f)} (\vec q, \dot {\vec q} ) = T(\dot {\vec q}) - V (\vec q) = \displaystyle\sum_{k = 1 }^{ f } \frac{1}{2}m \dot q_k - V(\vec q ) 
\end{gather*}
Con eq. de euler-lagrande 
\begin{gather*}
  \frac{d  }{d t } \frac{\partial L ^ {(f) } }{\partial \dot q_k } - \frac{\partial L ^ {(f) } }{\partial q_k } = 0  
\end{gather*}
Correspondientes a la segunda ley de newton $ m \ddot q_k = m \frac{d \dot q _k  }{d t } = - \frac{\partial V(\vec q ) }{\partial q_k }$. con los momento canónicamente conjugados 
\begin{gather*}
  p_k = \frac{\partial L ^ {(f) } }{\partial \dot q_k} 
\end{gather*}

En la formulación \textbf{Hamiltoniana } el sistema fisico conservativo con $ f  $ grados de liberta está descrito por 
\begin{gather*}
  H ^ {(f) }(\vec p , \vec q, \dot{\vec q }) = \displaystyle\sum_{k = 1 }^{f }p_k \dot q_k - L ^{(f) } (\vec q, \dot{\vec q })
\end{gather*}
Dado que $ p_k = \frac{\partial L ^ {(f) } }{\partial \dot q _k } $
\begin{gather*}
  H ^ {(f) }(\vec p , \vec q ) = T(\vec p ) + V(\vec q) = \displaystyle\sum_{k=1 }^{f } \frac{p_k ^2 }{2m } + V(\vec q ) 
\end{gather*}
Con ecuaciones de movimiento 
\begin{gather*}
  \dot q_k = \frac{\partial H ^ {(f) } }{\partial p_k }\qquad \qquad \qquad \dot p _k = - \frac{\partial H ^ {(f) } }{\partial q_k } 
\end{gather*}

\hfill 

\hfill 

\textbf{Teorema de Liouville } El agregado de puntos representativos que describe los puntos representativos en el espacio de fase se mueve como un fluido incompresible en el espacio de fase. 

\hfill 

\hfill 

Si la funcion densidad e probabilidad de estados en cualquier instante $ t  $ $ \rho(\vec p , \vec q ; t ) $ es una función conocida en el espacio de fase, que satisface 
\begin{gather*}
  \int_{}^{} \rho(\vec p, \vec q ) d\vec p d \vec q = 1  
\end{gather*}

\hfill 

\hfill 

Ecuación de \textbf{Liouville }
\begin{gather*}
  \frac{\partial \rho(\vec p, \vec q ; t ) }{\partial t } = L \rho(\vec p , \vec q ; t ) = \{H ^ {N }, \rho\}
\end{gather*}

\hfill 

\hfill 

\textbf{Mecánica estadistica en equilibrio: }Corresponde al caso $ \rho = \rho(\vec p, \vec q ) $
\begin{gather*}
  \frac{\partial \rho }{\partial t } = \{H ^ {(N) }, \rho\} = 0
\end{gather*}

\textbf{Mecánica estadística fuera de equilibrio } corresponde al caso $ \rho = \rho(\vec p, \vec q ; t ) $
\begin{gather*}
  \frac{\partial \rho }{\partial r } = \{H ^ {(N) }, \rho\} \neq 0 
\end{gather*}

\end{document}
