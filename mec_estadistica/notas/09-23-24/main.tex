
\documentclass{article}

\usepackage[most]{tcolorbox}
\usepackage{physics}
\usepackage{graphicx}
\usepackage{float}
\usepackage{amsmath}
\usepackage{amssymb}


\usepackage[utf8]{inputenc}
\usepackage[a4paper, margin=1in]{geometry} % Controla los márgenes
\usepackage{titling}

\title{Clase Mecánica Estadística }
\author{Manuel Garcia.}
\date{\today}

\renewcommand{\maketitlehooka}{%
  \centering
  \vspace*{0.05cm} % Espacio vertical antes del título
}

\renewcommand{\maketitlehookd}{%
  \vspace*{2cm} % Espacio vertical después de la fecha
}

\newcommand{\caja}[3]{%
  \begin{tcolorbox}[colback=#1!5!white,colframe=#1!25!black,title=#2]
    #3
  \end{tcolorbox}%
}

\begin{document}
\maketitle

\section{Ejercicios }
\subsection{Ejemplo 1 }

\textbf{a) } para un gas ideal de bosones, demuestre que la entropía $ S^B  $ se puede expresar como 
\begin{gather*}
  S^B = k_B \displaystyle\sum_{i }^{}\left[(n_i ^ {BE } + 1 ) \ln (1 + n_i^{BE }) - n_i ^ {BE } \ln n_i ^ {BE }\right] 
\end{gather*}
Donde $ n_i ^ {BE } $ es la distribución de Bose-Einstein, tenga en cuenta que la entropía de este sistema se puede expresar como $ S^B = k_B (\beta E+ \ln Z^B ) $ siendo $ Z ^B  $ la función de partición para el gas de bosones 

\textbf{b) } Muestre que en el limite de altas temperatura la entropía $ S ^B  $ tiende a la entropía de Gibbs -Boltzmann y que la función de gran partición para el gas de bosones definida como 
\begin{gather*}
  \ln Z^B = - \beta\mu N + \displaystyle\sum_{i }^{} \ln Z_i ^ {BE } 
\end{gather*}
Siendo $ Z_i ^ {BE } $ la función de Bose-Einstein, tiende a la función de partición de Gibbs- Boltzmann 

\textbf{Solución }

\textbf{a) } Se tiene que 
\begin{gather*}
  \ln Z^B = - \beta\mu N + \displaystyle\sum_{i }^{} \ln Z_i ^ {BE } = - \beta\mu N - \displaystyle\sum_{i }^{} \ln (1 - e ^ {- \beta(\epsilon_i - \mu)})
\end{gather*}
Dado que 
\begin{gather*}
  n_i ^ {BE } = \frac{1}{e ^ {\beta(\epsilon_i - \mu)} - 1 } 
\end{gather*}
Entonces 
\begin{gather*}
  n_i ^ {BE } e ^ {\beta(\epsilon_i - \mu)} - n_i ^ {BE } - n_i ^ {BE } = 1 \\
  n_i ^ {BE } e ^ {\beta(\epsilon_i - \mu)} - n_i ^ {BE } = 1 + n_i ^ {BE } \\
  \ln (n_i ^ {BE } e ^ {\beta(\epsilon_i - \mu)} - n_i ^ {BE } ) = \ln (1 + n_i ^ {BE }) \\
  \ln n_i ^ {BE }  + \ln (e ^ {\beta(\epsilon_i - \mu)} - n_i ^ {BE }) = \ln (1 + n_i ^ {BE }) \\
  \beta(\epsilon_i - \mu) = \ln \frac{1 + n_i^{BE }}{n_i^{BE }} \\
  \beta\epsilon_i = \beta \mu + \ln \frac{1 + n_i^{BE }}{n_i^{BE }}
\end{gather*}
Sustituyendo esto en la primera ecuación 
\begin{gather*}
  \ln Z^B = \beta\mu N - \displaystyle\sum_{i }^{} \ln \left(1 - \frac{n_i ^ {BE }}{1 + n_i ^ {BE }}\right) = - \beta\mu N + \displaystyle\sum_{i }^{} \ln (1 + n_i ^ {BE }) 
\end{gather*}
La energía interna promedio 
\begin{gather*}
  \bar \epsilon  = \displaystyle\sum_{i }^{} \epsilon_i n_i ^ {BE } \rightarrow \beta\bar\epsilon_i = \displaystyle\sum_{i }^{} (\beta\epsilon_i ) n_i ^ {BE }
\end{gather*}
Por lo tanto 
\begin{gather*}
  \beta\bar \epsilon = \displaystyle\sum_{i }^{} \left(\beta\mu + \ln \frac{1 + n_i to
  BE }{n_i ^ {BE }}\right)n_i ^ {BE }\\
  = \displaystyle\sum_{i }^{} \left[\beta\mu n_i ^ {BE } + n_i ^ {BE } \ln (1 + n_i ^ {BE }) - n_i ^ {BE }\ln n_i ^ {BE }\right]
\end{gather*}
Sustituyendo esto en la entropía $ S^B $ 
\begin{gather*}
  S^B = k_B \left(\displaystyle\sum_{i }^{} (\beta\mu n_i ^ {BE } + n_i ^ {BE } \ln ( 1 + n_i ^ {BE }) - n_i ^ {BE } \ln n_i ^ {BE }) - k_B (\displaystyle\sum_{i }^{} \beta\mu n_i ^ {BE } + \ln (1 + n_i ^ {BE }))\right) \\
  = k_b \displaystyle\sum_{i }^{} ( n_i ^ {BE } + 1 ) \ln ( 1 + n_i ^ {BE }) - n_i ^ {BE } \ln n_i ^ {BE }
\end{gather*}
Si $ T \rightarrow \infty; \ \ \beta = \frac{1}{k_B T } \rightarrow 0 ; \ \ n_i ^ {BE } = \frac{1}{e ^ {\beta(\epsilon_i - \mu)} - 1 } $
\begin{gather*}
  n_i ^ {BE } = \frac{1}{e ^ {\beta(\epsilon_i - \mu) } - 1 } \ \ \rightarrow \ \ n_i ^ {BE } \approx \frac{1}{e ^ {\beta\epsilon_i }} \rightarrow 0 ; \qquad \beta\epsilon_i \gg 0 
\end{gather*}
Teniendo en cuenta que $ \mu<\epsilon_0  \ \rightarrow \ \epsilon_i \gg \mu $ 
\begin{gather*}
  \beta\epsilon_i = \frac{\epsilon_i }{k_B T }; \ \ \epsilon_i \gg k_B T  \qquad \quad e ^ {\beta(\epsilon_i - \mu)} \approx e ^ {\beta\epsilon_i } \gg 0
\end{gather*}
Luego 
\begin{gather*}
  \ln(1 + n_i ^ {BE }) \approx \ln(1 ) \rightarrow 0 \\
  (n_i ^ {BE }) \ln (1 + n_i ^ {BE }) \approx 1 \ln 1 \rightarrow 0 
\end{gather*}
Así 
\begin{gather*}
  S^B = - k_B \displaystyle\sum_{i }^{} n_i ^ {BE }  \ln n_i ^ {BE }\\
  S = - k_B \displaystyle\sum_{i }^{} D_i \ln P_i 
\end{gather*}

\hfill 

\hfill

\textbf{b) } 
\begin{gather*}
  \ln Z^B = - \beta\mu N + \displaystyle\sum_{i }^{} \ln Z_i ^ {BE } = - \beta\mu N - \displaystyle\sum_{i }^{} \ln ( 1 - e ^ {- \beta(\epsilon_i - \mu)}) 
\end{gather*}
Si $ T \rightarrow  \infty $, $ \beta \rightarrow 0  \ \rightarrow \ - \beta\mu N \rightarrow 0 $\\
Así 
\begin{gather*}
  \ln Z^B = - \displaystyle\sum_{i }^{} \ln ( 1 - e ^ {- \beta(\epsilon_i - \mu)}) 
\end{gather*}
Dado que $ \epsilon_i \gg \epsilon_0 ; \ \ \mu<\epsilon_0  $\\
Así $ \epsilon_i \gg \mu  $\\
Luego $ e ^ {- \beta(\epsilon_i - \mu)} \approx e ^ {- \beta \epsilon_i } $ \\
Así $ \ln (1 - e ^ {- \beta\epsilon_i }) \approx e ^ {- \beta\epsilon_i } $
\begin{gather*}
  \ln Z^B = - \displaystyle\sum_{i }^{} \ln (1 - e ^ {-\beta \epsilon_i }) = \displaystyle\sum_{i }^{} e ^ {- \beta\epsilon_i } = Q_1  
\end{gather*}





\subsection{Ejemplo 2 }
La fluctuación relativa en el numero de partículas $ F  $ para el sistema descrito por el ensamble grancanónico se puede escribir como 
\begin{gather*}
  F = \frac{1}{N } \sqrt{\frac{1}{\beta} \frac{\partial \expval{N } }{\partial \mu }}  
\end{gather*}
\textbf{a) } Demuestre que $ F  $ en el numero de fermiones para un gas de N fermiones es 
\begin{gather*}
  F ^ {FD } = \sqrt{\frac{1 - n ^ {FD }}{n ^ {FD }}}  
\end{gather*}
\textbf{b) } Encuentre $ F ^ {FD } $ a $ T = 0  $

\hfill 

\textbf{a) } 
\begin{gather*}
  \expval{N } = N = n_i ^ {FD } = \frac{1}{e ^ {\beta( \epsilon_i - \mu )} + 1 } 
\end{gather*}
\begin{gather*}
  F_i ^ {FD } = \frac{1}{N } \sqrt{\frac{1}{\beta} \frac{\partial \expval{N_i } }{\partial \mu}}  
\end{gather*}
Así 
\begin{gather*}
  \frac{\partial \expval{N_i } }{\partial \mu} = \frac{(- 1) (-\beta) e ^ {\beta(\epsilon_i - \mu)}}{(e ^ {\beta(\epsilon_i - \mu)} + 1 )^2 }\\
  = \beta \left(\frac{1}{e ^ {\beta(\epsilon_i - \mu + 1 )} + 1 } - \frac{1}{(e ^ {\beta(\epsilon_i - \mu) } + 1 )^2}\right) \\
  = \beta(n_i ^ {FD } - (n_i ^ {FD})^2) \\
  = \beta n_i ^ {FD } (1 - n_i ^ {FD })
\end{gather*}
Sustituyendo esto en $ F_i ^ {FD } $
\begin{gather*}
  F_i ^ {FD } = \frac{\sqrt{\frac{1}{\beta} \beta n_i ^ {FD }( 1 - n_i ^ {FD })} }{\sqrt{(n_i ^ {FD})^2} } \\
  F_i ^ {FD } = \sqrt{\frac{1 - n_i ^ {FD }}{n_i ^ {FD }}} 
\end{gather*}

\hfill 

\hfill

\textbf{b) } Si $ T \rightarrow 0 = n ^ {FD } = 1  $
\begin{gather*}
  \rightarrow F_i ^ {FD } = \sqrt{\frac{1 - 1 }{1}} = 0  
\end{gather*}






\subsection{Ejemplo 3 }
Para un gas ideal de $ N  $ partículas simples sin espín contenido en un recipiente de volumen $ V  $ a temperatura $ T  $ se satisface las siguientes ecuaciones de estado 
\begin{gather*}
  E = \frac{3 }{2 } N k_B T \\
  PV = \frac{2}{3} E 
\end{gather*}
Ahora considere que las partículas del gas ideal puede ser función de spin $ S = \frac{1}{2} $ o bosones de espín $ S = 1  $; de tal forma que los fermiones tienen la misma masa $ m  $ que los bosones, cumpliéndose que cuando el gas ideal es de fermiones, la energía interna y presión son respectivamente 
\begin{gather*}
  E_B \quad \text{ y } \quad p_F  
\end{gather*}
encontrar que si el gas ideas es de bosones, la energía y la presión son 
\begin{gather*}
  E_B \quad \text{ y }\quad P_B 
\end{gather*}
Haciendo uso de la expresión semiclásica, en la que 
\begin{gather*}
  \lambda = \frac{h }{(2 \pi m k_B T )^ {1/2 }} 
\end{gather*}
Para dos gases ideales, el primero de $ N  $ fermiones y el segundo de $ N  $ bosones, contenido en un recipiente de volumen $ V  $ y a la misma temperatura $ T  $, define la diferencia de energía interna 
\begin{gather*}
  \Delta E = E_F - E_B  
\end{gather*}
 y la diferencia de presión 
 \begin{gather*}
  \Delta P = P_F - P_B  
 \end{gather*}
 Entre los dos gases.

 En la aproximación analítica tiene que la energía interna en un gas de partícula en espín $ S  $ es 
 \begin{gather*}
  E = \frac{3 }{2} Nk_B T \left[1 + \frac{1 }{2 ^ {5/2 }} \frac{N \lambda^3 }{(2S + 1 )V }\right] 
 \end{gather*}
 Siendo $ (+ ) $ para fermiones $ (S = \frac{1}{2}) $, $ (- )  $ para bosones $ (S = 1 ) $ para este gas ideal se cumple que 
 \begin{gather*}
  PV = \frac{2 }{3 } E  
 \end{gather*}
 Para gas de fermiones 
 \begin{gather*}
  E_F = \frac{3 }{2 } Nk_B T \left[ 1 + \frac{1}{2 ^ {3/2 }} \frac{1}{(2 \cdot \frac{1}{2} + 1 )}(N/V ) \lambda^3 \right] \\
  E_B = \frac{3 }{2 } Nk_B T \left[ 1 + \frac{1}{2 ^ {3/2 }} \frac{1}{(2 \cdot 1 + 1 )}(N/V ) \lambda^3 \right] 
 \end{gather*}
 Así 
 \begin{gather*}
  \Delta E = E_F - E_B = \frac{3 }{2 } k_B T \left[\frac{1}{\sqrt{32 } } \frac{1}{2} (N/V)\lambda^3 + \frac{1}{32 }\frac{1}{3} (N/V)\lambda^3 \right] = \frac{3 }{2 } Nk_B T \frac{5}{33.94 } (N/V ) \lambda^3 \\
  \Delta E = 0.044 N k_B T (N/V) \lambda^3 
 \end{gather*}

 \hfill 

 \hfill 

 \begin{gather*}
  \Delta P = P_F  - P_B = \frac{1}{V } \left(\frac{2}{3} E_F - \frac{2}{3 } E_B \right) = \frac{1}{V } \frac{2}{3} \Delta E \\
  \Delta P = 0.029 k_B T (N/V)^2 \lambda^3 
 \end{gather*}

 \hfill 

 \hfill 

 \textbf{b) } Si $ N/2  $ sea fermiones y $ N/2  $ sea bosones, determine $ E  $ y $ P  $ 
 \begin{gather*}
  E ^ {\text{gas y espin }} = E_P + E_B = \frac{3}{2} (N/2) k_B T \left[1 + \frac{1}{\sqrt{32 } }\frac{1}{2} \frac{1}{2} (N/V) \lambda^3 \right] + \frac{3}{2} (N/2) k_B T \left[1 + \frac{1}{\sqrt{32 } }\frac{1}{3} \frac{1}{2} (N/V) \lambda^3 \right]
\end{gather*}
La diferencia con respecto al gas ideal de particulas sin espin 
\begin{gather*}
  \Delta E = E ^ {espin } - E ^ {sin\ espin } = 0.011 N k_B T (N/V) \lambda^3  \\
  \Delta P = P ^ {espin } - P ^ {sin\ espin } = \frac{1}{V } \left(\frac{2}{3 } E ^ {espin } - \frac{2}{3} E ^ {sin\ espin }\right) = \frac{1}{V} \frac{2}{3} \Delta E  \\
  \Delta P = 0.007 k_B T (N/V)^2 \lambda^3 
\end{gather*}

\hfill 

\hfill 

\begin{gather*}
  T \lambda^3  = T \frac{1}{(T ^ {1/2 })^3 } = T ^ {-1/2}\\
  \Delta E \approx \frac{1}{\sqrt{T } } \qquad \qquad \Delta P \approx \frac{1}{\sqrt{T} }
\end{gather*}

\end{document}
