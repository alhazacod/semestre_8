
\documentclass{article}

\usepackage[most]{tcolorbox}
\usepackage{physics}
\usepackage{graphicx}
\usepackage{float}
\usepackage{amsmath}
\usepackage{amssymb}
\usepackage{gensymb}


\usepackage[utf8]{inputenc}
\usepackage[a4paper, margin=1in]{geometry} % Controla los márgenes
\usepackage{titling}

\title{Lista de Ejercicios I }
\author{Manuel Garcia.}
\date{\today}

\renewcommand{\maketitlehooka}{%
  \centering
  \vspace*{0.05cm} % Espacio vertical antes del título
}

\renewcommand{\maketitlehookd}{%
  \vspace*{2cm} % Espacio vertical después de la fecha
}

\newcommand{\caja}[3]{%
  \begin{tcolorbox}[colback=#1!5!white,colframe=#1!25!black,title=#2]
    #3
  \end{tcolorbox}%
}

\begin{document}
\textbf{Lista de Ejercicios II }\hfill \textit{Manuel Angel Garcia M.}

\section{}
\begin{gather*}
  m - M = 5(\log{r} - 1)
\end{gather*}
\begin{itemize}
  \item \textbf{a) } $ m = - 30.32, \ M = 1.25  $ 
    \begin{gather**}
      \text{Despejando }r \\
      r = 10 ^ {\frac{m - M }{5 } + 1 } \\
      \text{Reemplazando los valores de }m \text{ y } M \\
      r = 10 ^ {1 - \frac{31.57}{5}} \approx 4.85 \times10^{-6} \text{pcs}\\
      r = (4.85 \times10^{-6}) (3.262) = 1.58 \times10^{-5} \text{A-L}
    \end{gather**}
  \item \textbf{b) } $ B-V = 0.55 $ 
    \begin{gather*}
      B - V = 0.55 = -2.5 \log{\frac{F_B}{F_V}} \quad \rightarrow \quad \frac{F_B }{F_v } = 10 ^ {\frac{0.55 }{-2.5 }}
    \end{gather*}
\end{itemize}



\section{}
Paralaje anual de $ 0.5'' $

\begin{itemize}
  \item \textbf{a) } 
    \begin{gather*}
      p = 0.5'', \qquad d = \frac{1}{p } \\
      d = \frac{1}{1/2 } = 2 \text{pcs } = 412,53 \text{UA }
    \end{gather*}
  \item \textbf{b) } 
    \begin{gather*}
      d_1 = 2 \text{pcs }\quad d_2 = \frac{1 }{1 }= 1 \text{pcs } \\
      \frac{d_1 }{d_2 } = \frac{2}{1} \quad \rightarrow \quad d_1 = 2 d_2 
    \end{gather*}
    $ d_2  $ es la mitad de $ d_1  $, por lo tanto $ d_1 > d_2  $.
  \item \textbf{c) } Si $ M_1 = M_2 = M = -2.0  $
    \begin{gather*}
      \text{Para la estrella 1 }, \quad m_1 = - 2 + 5 \log{2} - 5 \approx - 5.495 \\
      \text{Para la estrella 2 }, \quad m _2 = -1 + 5 \log{1 } - 5 = -7.0
    \end{gather*}
    Podemos observar que $ m_2  $ es menor en magnitud aparente.
\end{itemize}





\section{}
$ d = \frac{1 }{0.0001 } = 10000 \text{pcs }$ por lo tanto sí se puede diferenciar una distancia de $ 10000 \text{pcs} $.





\section{}
Estrella a $ 690 \text{kpc} $ con $ M = 5  $, al explotar su brillo se incrementa $ 10 ^ {10 } $ veces la original.

\begin{gather*}
  m = M + 5 \log{d } - 5 \quad \rightarrow \quad m = 5 \log{690000} = 29.2 \\
  \text{La nueva magnitud aparente cuando el brillo incremente $ 10^{10}  $ veces la original}\\
  m _{\text{supernova}} = m - 2.5 \log{10 ^ {10 } } = 29.2 - 25 = 4.2
\end{gather*}



\section{}
\begin{itemize}
  \item \textbf{a) } 
    \begin{gather*}
      f = \frac{c }{\lambda} = \frac{3 \times10^{8 }}{4.25 \times10^{-7 }} = 0.706 \times10^{15 } Hz \\
      k = \frac{2\pi }{\lambda} = \frac{2 \pi }{4.25 \times10^{-7 }} = 148 \times10^{7 } 1/m
    \end{gather*}
    Pertenece al espectro visible (entre 400nm y 700nm).
  \item \textbf{b) } su velocidad de propagacion se reduce a $ \frac{4c }{5 } $
    \begin{gather*}
      n = \frac{c }{ v } = \frac{c }{4/5 c } = \frac{5}{4} = 1.25 
    \end{gather*}
  \item \textbf{c) } 
    \begin{gather*}
      \text{Si } f = 7.06 \times10^{14 } Hz \quad \text{, }\quad v = \frac{5 }{4 } c \\
      \lambda = \frac{v}{f} = \frac{\frac{4}{5 } (3 \times10^{8 })}{7.06 } = 3.4 \times10^{-7 }m = 340 nm
    \end{gather*}
\end{itemize}




\section{}
\begin{gather*}
  E = \frac{hc }{\lambda}, \qquad c = 3 \times10^{8} m/s, \qquad h = 6.626 \times10^{-34 }Js
\end{gather*}
\begin{itemize}
  \item Para el sol $ \lambda = 500 \times10^{-9 }m  $
    \begin{gather*}
      E _{sol } = \frac{(6.626 \times10^{-34 }Js )(4 \times10^{8 }m/s )}{500 \times10^{-9 }} = 3.98 \times10^{-19 }J \\
      E _{sol }  = \frac{3.98 \times10^{-19 }J\ eV }{1.60 \times10^{-19 }J } = 2.48 eV
    \end{gather*}

  \item Para Sirio $ \lambda = 300 \times10^{-9 } $ 
    \begin{gather*}
      E _{sirio } = \frac{(6.626 \times10^{-34 }Js)(3 \times10^{8 }m/s)}{300 \times10^{-9 }m  } = 6.62 \times10^{-19 }J = 4.14 eV 
    \end{gather*}

  \item Para Betelgeuse $ \lambda = 900 \times10^{-9 } $
    \begin{gather*}
      E _{Be } = \frac{(6.626 \times10^{-34 }Js ) (3 \times10^{8 } m/s ) }{900 \times10^{-9 }m } = 1.38 eV
    \end{gather*}
\end{itemize} 




\section{}
De la expresion en parsecs tenemos que $ m - M = 5(\log{r} - 1 ) $, aplicando que $ r = R \times10^{6 } $ tenemos $ m - M = 5(\log{R \times10^{6 }}- 1 ) = 5 (\log{R } + \log{10^6}- 1) \rightarrow m - M = 5(\log{R} + 5)$


\end{document}
