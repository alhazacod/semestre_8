
\documentclass[12pt]{article}
\usepackage{amsmath}
\usepackage{amsfonts}
\usepackage{amssymb}
\usepackage{graphicx}

\title{Resumen de la Métrica de Schwarzschild y Geodésicas}
\author{Manuel Angel Garcia, Carlos Andres Llanos \\
Facultad de Física, Universidad Nacional de Colombia}
\date{}

\begin{document}
\maketitle

\section*{Introducción}
El documento examina la métrica de Schwarzschild, una solución de las ecuaciones de campo de Einstein en la relatividad general, que describe el espacio-tiempo alrededor de una masa esférica no cargada y no rotante. La derivación de esta métrica parte de la métrica de Minkowski en coordenadas esféricas y se modifica introduciendo funciones radiales. Se calcula el tensor de Ricci y se resuelven las ecuaciones de campo de Einstein en el vacío para encontrar la métrica de Schwarzschild en su forma final, que incluye el radio de Schwarzschild $R_s$.

Se analiza la ecuación de la geodésica, que describe el movimiento de las partículas en este espacio-tiempo, así como las órbitas de las partículas y la luz. La ecuación diferencial resultante para las órbitas introduce el parámetro de impacto $b$, que determina si una órbita es estable, inestable o si la partícula caerá hacia el centro. Estos resultados nos ayudan a entender el comportamiento de objetos en campos gravitacionales intensos, los alrededores de un agujero negro clasico.

\end{document}
