\documentclass{article}

\usepackage[most]{tcolorbox}
\usepackage{physics}
\usepackage{graphicx}
\usepackage{float}
\usepackage{amsmath}
\usepackage{amssymb}
\usepackage{gensymb}


\usepackage[utf8]{inputenc}
\usepackage[a4paper, margin=1in]{geometry} % Controla los márgenes
\usepackage{titling}

\title{Lista de Ejercicios I }
\author{Manuel Garcia.}
\date{\today}

\renewcommand{\maketitlehooka}{%
  \centering
  \vspace*{0.05cm} % Espacio vertical antes del título
}

\renewcommand{\maketitlehookd}{%
  \vspace*{2cm} % Espacio vertical después de la fecha
}

\newcommand{\caja}[3]{%
  \begin{tcolorbox}[colback=#1!5!white,colframe=#1!25!black,title=#2]
    #3
  \end{tcolorbox}%
}

\begin{document}
\textbf{Lista de Ejercicios I }\hfill \textit{Manuel Angel Garcia M.}

\section{}

\hfill 

\textbf{(a)}
\begin{gather*}
  \hat r = \rho \sin{\phi } \cos{\theta} \hat i + \rho \sin{\phi } \sin{\theta} \hat j + \rho \cos{\phi } \hat k 
\end{gather*}

\textbf{(b)}
\begin{gather*}
  \alpha = 90\degree - \phi \qquad \qquad \beta = 90\degree - \theta \\
  \text{Despejando }\theta \text{ y } \phi\\
  \phi = \frac{\pi}{2} - \alpha \qquad \qquad \theta = \frac{\pi}{2} - \beta \\
  \sin{\frac{\pi}{2} - \alpha} = \cos{\alpha} \qquad \qquad \cos{\frac{\pi}{2} - \alpha} = \sin{\alpha}\\
  \sin{\frac{\pi}{2} - \beta} = \cos{\beta} \qquad \qquad \cos{\frac{\pi}{2} - \beta} = \sin{\beta}\\
  \text{Reemplazando en }\hat r \\
  \hat r = \rho \cos{\alpha} \sin{\beta} \hat i + \rho \cos{\alpha} \cos{\beta} \hat j + \rho \sin{\alpha} \hat k
\end{gather*}

\textbf{(c) } 

\hfill

\textit{Coordenadas horizontales:} Tomando el plano del horizonte como el plano $ xy  $ y el eje cenit como el eje z positivo; $ \beta $ seria la coordenada acimut y $ \alpha $ seria la altura.

\hfill

\textit{Coordenadas ecuatoriales geocentricas:} Tomando el plano $ xy  $ como el plano ecuatorial y el eje $ z  $ positivo como el norte celeste, $ \theta  $ seria la ascension recta y $ \alpha $ seria la declinacion.

\hfill

\textit{Coordenadas eclipticas:} Tomando el plano $ xy  $ como el plano ecliptico y el eje $ z  $ positivo como el polo ecliptico norte; $ \theta  $ seria la longitud y $ \alpha $ seria la latitud.

\hfill

\textit{Coordenadas galácticas:} Tomando el plano $ xy  $ como el plano del circulo galáctico y el eje $ z  $ positivo como el polo norte galáctico; $ \theta  $ seria la longitud galactica y $ \alpha $ seria la latitud galactica.


\section{}

\hfill 

\textbf{(a)} 
\begin{gather*}
  m r_A v_A = m r_B v_B \quad \rightarrow \quad \frac{v_A }{v_B } = \frac{r_B }{r_A } \\
  \text{Como } r_B = 4r_A  \\ 
  \frac{v_A }{v_B } = \frac{4 r_A }{r_A } = 4 \quad \rightarrow \quad v_A = 4v_B
\end{gather*}

\hfill 

\textbf{(b)} 
\begin{gather*}
  \frac{T_A^2 }{r_A^3 } = \frac{T_B^2 }{r_B^3 } \quad \rightarrow \quad \left(\frac{T_A }{T_B }\right) ^2 = \left(\frac{r_A }{r_B }\right) ^ {3 } \quad \rightarrow \quad  \frac{T_A }{T_B }  = \left(\frac{r_A }{r_B }\right) ^ {3/2 } \\ 
  \frac{T_A }{T_B }  = \left(\frac{r_A }{4r_A }\right) ^ {3/2 } = \frac{1}{8} \\
  T_B = 8T_A
\end{gather*}

\hfill 

\textbf{(c) } El potencial gravitacional es $  u = - \frac{GMm }{r } $ y para orbitas circulares tenemos que $ v^2 = \frac{GM }{r } $
\begin{gather*}
  H = k+u = \frac{GMm }{2r } - \frac{GMm }{r } = - \frac{GMm }{r } \\
  H_A = -\frac{GMm }{r_A }  \qquad \qquad \qquad H_B = - \frac{GMm }{r_B } = - \frac{GMm }{4r_A}
\end{gather*}
\begin{gather*}
  \frac{H_A }{H_B } = \frac{\frac{GMm }{r_A }}{\frac{GMm }{4r_A }} =4\\
  H_A = 4H_B
\end{gather*}
El planeta $ A  $ tiene 4 veces la energia total del planeta $ B  $.

\section{}

\hfill 

\textbf{(a)} 
\begin{gather*}
  m r_a v_a = m r_p v_p \qquad \qquad \frac{r_a }{r_p } = \frac{v_p }{v_a }\\
  \text{Como } v_a = \frac{v_p }{3 }\\
  \frac{r_a }{r_p } = \frac{v_p }{v_p/3 } = 3 \\
  r_a = 3 r_p 
\end{gather*}

\hfill 

\textbf{(b)} 

\hfill 

\textit{Semieje Mayor} 
\begin{gather*}
  a = \frac{r_a + r_p }{2 } = \frac{4 }{2} r_p = 2r_p
\end{gather*}

\textit{Semidistancia Focal }
\begin{gather*}
  c = \frac{r_a - r_p }{2 } = \frac{2r_p }{2} = r_p  
\end{gather*}

\textit{Excentricidad }
\begin{gather*}
  \epsilon = \frac{c }{a } = \frac{r_p }{2r_p } = \frac{1}{2}
\end{gather*}

\hfill 

\textbf{(c) } 
\begin{gather*}
  U_a = - \frac{GMm }{r_a } \qquad \qquad U_p = - \frac{GMm }{r_p } \\ 
  \frac{U_a }{U_p } = \frac{- \frac{GMm }{r_a }}{- \frac{GMm }{r_p }} = \frac{r_p }{r_a } = \frac{1}{3} \quad \rightarrow \quad U_p = 3 U_a
\end{gather*}



\section{}

\hfill 

\textbf{(a)}

\hfill 

\textit{Semieje Mayor} 
\begin{gather*}
  a = \frac{r_A + r_B }{2 } = \frac{(8+2)10^8 }{2} = 5\times10^8
\end{gather*}

\textit{Semidistancia Focal }
\begin{gather*}
  c = \frac{r_a - r_p }{2 } = \frac{(8+2)10^8 }{2} = 5 \times 10^8
\end{gather*}

\textit{Excentricidad }
\begin{gather*}
  \epsilon = \frac{c }{a } = \frac{3 \times 10^8 }{5 \times 10^8 } = \frac{3 }{5 }
\end{gather*}

\hfill 

\textbf{(b)} Tenemos que $ a_2^3 = 125\times 10 ^ {24 }  $, $ a_1^3 = 8 \times 10 ^ {24}  $ y $ T_1 = 2  $años. Usando la tercera ley de kepler. 
\begin{gather*}
  \frac{T_1^2 }{a_1^3 } = \frac{T_2^2 }{a_2^3 } \quad \rightarrow \quad T_2 = \sqrt{\frac{T_1^2 a_2^3 }{a_1^3 }}  \\
  T_2 = \sqrt{\frac{2^2 (125\times 10 ^ {24 })}{8 \times 10 ^ {24 }}} \text{año} = 5 \sqrt{5/2} \text{ año} 
\end{gather*}

\hfill 

\textbf{(c) } En el periastro $ v_1 = v_2  $. 
\begin{gather*}
  \omega = \frac{2\pi}{T} = \pi \frac{1}{\text{año}} \\
  v_{p2} = v_{p1} = \pi \frac{1}{\text{año}} 2 \times 10^8 \text{km } = 2\pi\times10^{8} \frac{\text{km}}{\text{año}} \\
  v _{p2} =2\pi \times10^{8 } \frac{\text{km}}{\text{año}} \frac{\text{año}}{3.15 \times10^{7 } s } = \frac{20\pi }{3.15 } \frac{\text{km}}{\text{s}} \\
  v _{p2}  =2\pi \times10^{8 } \frac{\text{km}}{\text{año}} \frac{6.68 \times10^{-9 }\text{UA}}{ \text{km} } = (2\pi)(6.68)\times10^{-1 } \frac{\text{UA}}{\text{año}} 
\end{gather*}
En el apoastro
\begin{gather*}
  m v _{a2 } r_a = m v _{p2 } r_p \quad \rightarrow \quad v _{a2 } = \frac{v _{p2 } r_p }{r_a } = \frac{v _{p2 } (2 \times10^{8 })}{8 \times10^{8}} = \frac{v _{p2 } }{4} \\
  v _{a2 } = \frac{1}{4}\left(\frac{20\pi}{3.15 }\right) \frac{\text{km}}{\text{s}}
\end{gather*}

\hfill 

\textbf{(d)} Para orbitas circulares y $ m\ll M $ tenemos que $ GM T^2 = r^3 4 \pi^2 $.
\begin{gather*}
  M = \frac{r_1^3 4\pi^2 }{G T_1^2} = \frac{(2 \times10^{11 }\text{m})^3 4\pi^2 }{(6.67 \times10^{-11 }\frac{m^3 }{\text{kg } \text{s}^2})(6.31 \times10^{7 } \text{s})^2} = 1.19 \times10^{30 }\text{kg} \\
  M = 1.19 \times10^{30 }\text{kg} \frac{M _{\odot} }{1.989 \times10^{30} \text{kg }} = 0.598 M _{\odot } 
\end{gather*}

\hfill 

\section{}
Tenemos que $ M_T = 2.14 \times10^{22 }\text{kg } $ y $ R_T = 1.35 \times10^{3 } \text{km } $.
\begin{gather*}
  g_T = \frac{GM_T}{R_T^2 } = \frac{(6.67 \times10^{-11 }\frac{m^3 }{\text{kg } \text{s}^2})(2.14 \times10^{22 }\text{kg })}{(1.35 \times10^{6 } \text{m })^2} = 7.83 \times10^{-1 } \frac{\text{m}}{\text{s}^2 }
\end{gather*}

Para hallar la masa ($ M  $) de un planeta con el radio de la tierra ($ R = 6.38 \times10^{6 }m $) que tenga la misma aceleracion ($ g $) que el planeta triton debemos igualar ambas aceleraciones y despejar $ M  $.

\begin{gather*}
  g = \frac{GM }{R^2 } = g_T \\
  \frac{GM }{R^2 } = \frac{GM_T}{R_T^2} \quad \rightarrow \quad M = \frac{M_T R^2 }{R_T^2 } \\
  M = \frac{(2.14 \times10^{22 }\text{kg })(6.38 \times10^{6 }m)^2}{(1.35 \times10^{6 } \text{m })^2} = 4.78 \times10^{23} \text{kg }
\end{gather*}

De forma analoga para $ M = 5.97 \times10^{24 }\text{kg } $
\begin{gather}
  R = \sqrt{\frac{M R_T^2 }{M_T }} = \sqrt{\frac{(5.97 \times10^{24 }\text{kg })(1.35 \times10^{3 } \text{km })^2}{(2.14 \times10^{22 }\text{kg })}} = 22.55 \times10^{3 }\text{km}
\end{gather*}



\end{document}
