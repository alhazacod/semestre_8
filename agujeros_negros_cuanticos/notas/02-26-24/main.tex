\documentclass{article}

\usepackage[most]{tcolorbox}
\usepackage{physics}
\usepackage{graphicx}
\usepackage{float}
\usepackage{amsmath}
\usepackage{amssymb}


\usepackage[utf8]{inputenc}
\usepackage[a4paper, margin=1in]{geometry} % Controla los márgenes
\usepackage{titling}

\title{Clase 5 }
\author{Manuel Garcia.}
\date{\today}

\renewcommand{\maketitlehooka}{%
  \centering
  \vspace*{0.05cm} % Espacio vertical antes del título
}

\renewcommand{\maketitlehookd}{%
  \vspace*{2cm} % Espacio vertical después de la fecha
}

\newcommand{\caja}[3]{%
  \begin{tcolorbox}[colback=#1!5!white,colframe=#1!25!black,title=#2]
    #3
  \end{tcolorbox}%
}

\begin{document}
\maketitle

\section{Ensamble de Bosones Libres con Frecuencia $ \omega $ (para un grado de libertad)}
\textbf{Modelo complejo }
\begin{gather*}
  \hat H = \displaystyle\sum_{}^{}\hat H_i, \qquad \qquad Z = \Pi Z_i, \qquad \qquad \hat N = \displaystyle\sum_{}^{} \hat N_i  
\end{gather*}
Para un grado de libertad ($ \omega $)
\begin{gather*}
  \hat H = \omega \hat a ^\dagger \hat a
\end{gather*}
Su ecuacion de valores propios 
\begin{align*}
  \hat H \ket{n } &= E_n \ket{n } \\
                  &= n\omega \ket{n }
\end{align*}
$ \hat H  $ tambien se puede escribir $\qquad \qquad \qquad \hat H = \omega \hat N \qquad \text{Donde } \hat N := \hat a^\dagger \hat a  $ 

\hfill 

Entonces 
\begin{align*}
  \omega\hat N \ket{n } &= n \omega \ket{n }\\
  \hat N \ket{n } &= n \ket{n }
\end{align*}
$ \{\ket{n }\} $ base para el espacio de Hilbert asociado 
\begin{gather*}
  \bra{n }\ket{n' } = \delta _{n n' } \qquad \qquad \sum_{}^{} \ket{n } \bra{n } = 1  
\end{gather*}
Vamos a usar la funcion de perticion $ Z = \frac{1}{1 - e ^ {-\beta \omega}} = \displaystyle\sum_{n }^{} e ^ {- \beta n \omega}$.




\section{Reglas de Conmutación }
\begin{gather*}
  \left[ \hat a , \hat a  \right] = \left[ \hat a^\dagger  , \hat a ^\dagger  \right] = 0 \\
  \left[ \hat a  , \hat a ^\dagger   \right] = 1 
\end{gather*}




\section{Representación de Fock} 

\hfill 

\textbf{Estado de vacío:} 
\begin{gather*}
  \hat a \ket{0 } := 0 
\end{gather*}

\textbf{Base } 
\begin{gather*}
  \ket{0 },\ \hat a ^ {\dagger } \ket{0 } , \ \cdots,\  \frac{1}{\sqrt{n! } }(\hat a ^ {\dagger })^n \ket{n }
\end{gather*}
donde 
\begin{gather*}
  \frac{1}{\sqrt{n! } }(\hat a ^ {\dagger })^n \ket{0 } = \ket{n } 
\end{gather*}



\end{document}
