\documentclass{article}

\usepackage[most]{tcolorbox}
\usepackage{physics}
\usepackage{graphicx}
\usepackage{float}
\usepackage{amsmath}
\usepackage{amssymb}


\usepackage[utf8]{inputenc}
\usepackage[a4paper, margin=1in]{geometry} % Controla los márgenes
\usepackage{titling}

\title{Clase Agujeros Negros Cuanticos }
\author{Manuel Garcia.}
\date{\today}

\renewcommand{\maketitlehooka}{%
  \centering
  \vspace*{0.05cm} % Espacio vertical antes del título
}

\renewcommand{\maketitlehookd}{%
  \vspace*{2cm} % Espacio vertical después de la fecha
}

\newcommand{\caja}[3]{%
  \begin{tcolorbox}[colback=#1!5!white,colframe=#1!25!black,title=#2]
    #3
  \end{tcolorbox}%
}

\begin{document}
\maketitle

\section{Campos sobre variedades curvas }

\hfill

\textbf{Metrica de Minkowski } $ ds^2 = -dt^2 + dx^2 + dy^2 + dz^2 \qquad  $ Global

\textbf{Metrica de Riddler } $ ds ^2 = - e ^ {2 a \xi } d\eta^2 + e ^ {2 a \xi }d\xi^2 + dy^2 + dz^2  \qquad  $ Local  

\textbf{Metrica de Kruskal }$ ds^2 = \frac{32M^3 }{r} e ^ {-\frac{r}{2M }}(-t\tau^2 + dz^2 ) + r^2(d\theta^2 + \sin^2{\theta}d\phi^2) \qquad $ 

\textbf{Metrica de Swarzschild } $ ds^2 = - \left(1 - \frac{2M }{ r }\right)dt^2 + \left(1 - \frac{2M }{r }\right)dr^2 + r^2 \left(d\theta^2 + \sin^2{\theta }d\phi^2 \right) $

Para dos observadores con la misma velocidad angular su relacion entre el sus tiempos es 
\begin{gather*}
  \frac{\Delta T_2 }{\Delta T_1 } = \sqrt{\frac{1 - \frac{2M }{r_2 }}{1 - \frac{2M }{r_1 }}}
\end{gather*}
\end{document}
