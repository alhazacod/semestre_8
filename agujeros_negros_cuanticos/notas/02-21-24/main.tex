\documentclass{article}

\usepackage[most]{tcolorbox}
\usepackage{physics}
\usepackage{graphicx}
\usepackage{float}
\usepackage{amsmath}
\usepackage{amssymb}


\usepackage[utf8]{inputenc}
\usepackage[a4paper, margin=1in]{geometry} % Controla los márgenes
\usepackage{titling}

\title{Clase 4 }
\author{Manuel Garcia.}
\date{\today}

\renewcommand{\maketitlehooka}{%
  \centering
  \vspace*{0.05cm} % Espacio vertical antes del título
}

\renewcommand{\maketitlehookd}{%
  \vspace*{2cm} % Espacio vertical después de la fecha
}

\newcommand{\caja}[3]{%
  \begin{tcolorbox}[colback=#1!5!white,colframe=#1!25!black,title=#2]
    #3
  \end{tcolorbox}%
}

\begin{document}
\maketitle

\section{Formalismo U-T }
De la clase pasada
\begin{gather*}
  \underset{\text{El vacio }}{\bra{0(\beta)}\hat A \ket{0(\beta)} } \overset{?}{=}<\hat A > = \frac{\sum_{n }^{} \bra{n }\hat A \ket{n } e ^ {-\beta E_n }   }{Z (\beta )} 
\end{gather*}
Teniamos que
\begin{gather*}
  \ket{0(\beta)} = \displaystyle\sum_{n }^{} f_n(\beta) \ket{n } 
\end{gather*}
donde $ \ket{n } $ son bases del espacio de Hilbert.

\hfill 

Operando con esto llegamos a 
\begin{gather*}
  f_n^*(\beta) f_m(\beta)= Z ^ {-1 }(\beta)\ \ \delta _{nm} e ^ {-\beta E_n } 
\end{gather*}

Cuando empezamos a hacer el desarrollo suposimos un espacio de Hilbert $ \mathcal H  $ con bases $ \{\ket{n }\} $. Ahora vamos a definir un nuevo espacio de hilbert compuesto por el espacio original y uno nuevo $ \bar{\mathcal H } = \mathcal H \otimes \tilde{\mathcal H}$. Ahora vamos a definir $ \ket{0(\beta)} \in \bar{\mathcal H } $, por lo tanto 
\begin{gather*}
  \ket{0(\beta)} = \displaystyle\sum_{n }^{} f_n(\beta) \ket{n }, \qquad \qquad \text{Donde }\ket{n }\in \mathcal H, \ f_n(\beta) \in \tilde{\mathcal H }  
\end{gather*}

\textbf{Definicion de }$ f_n(\beta)  $ \textbf{en }$ \tilde{\mathcal H } $
\begin{gather*}
  f_n(\beta) := e ^ {-\frac{1}{2} \beta E_n}Z ^ {-1/2 }(\beta) \ket{\tilde n } 
\end{gather*}

\begin{align*}
  \ket{0(\beta)} &= \displaystyle\sum_{n }^{}f_n(\beta)\ket{n } \\
             &= \displaystyle\sum_{n }^{} e ^ {-\frac{1}{2} \beta E_n}Z ^ {-1/2 }(\beta)\ \ \ket{\tilde n } \otimes \ket{n }\\
             &= Z ^ {-1/2 }(\beta)\displaystyle\sum_{n }^{} e ^ {-\frac{1}{2} \beta E_n}\ \ \ket{n,\ \tilde n }
\end{align*}

Por lo tanto 

\begin{align*}
  \bra{0(\beta)}\hat A \ket{0(\beta)} &= Z ^ {-1 }(\beta) \displaystyle\sum_{n,m }^{} e ^ {- \frac{\beta}{2}(E_n+E_m )} \bra{n }\vec A \ket{n }\delta _{nm }    \\
    &=Z ^ {-1 }(\beta) \displaystyle\sum_{n }^{} e ^ {- \beta E_n} \bra{n }\vec A \ket{n }
\end{align*}

\section{Interpretacion Física del formalismo U-T }
Notemos la necesidad de dos hamiltonianos $ \hat H , \ \hat{\tilde H } $, hay que modelar dos sistemas. Con esto se logra describir un sistema termico.  Esto se puede ver como un ensamble termico (el sistema a estudiar y el baño termico), esto se puede ver como el sistema a estudiar y el resto del universo.


Si dos subsistemas son idénticas copias uno del otro y están entrelazados (\textit{entangled}) cuanticamente 
\begin{gather*}
  \underset{\text{fase compartida}}{Z ^ {-1/2 }(\beta) \displaystyle\sum_{n }^{} e ^ {- \beta E_n}} \underset{\text{estado compuesto}\ \ket{n }\otimes \ket{\tilde n}}{\ket{n,\ \tilde n }}
\end{gather*}
de tal manera que conforman un estado puro para el sistema total, cada uno de ellos llega macroscópicamente indistingible de un cuerpo caliente a temperatura definida T.

\section{Vario de los Campos en el Formalismo U-T }
Campos para un grado de libertad. 

\textbf{Modelo: } Ensamble de bosones libres con frecuencia $ \omega $.

\hfill 

\hfill 

Descripcion de $ N  $ bosones libres no interactuantes en equilibrio térmico en términos de un grado de libertad.


\end{document}
