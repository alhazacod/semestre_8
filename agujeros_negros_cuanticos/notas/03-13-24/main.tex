\documentclass{article}

\usepackage[most]{tcolorbox}
\usepackage{physics}
\usepackage{graphicx}
\usepackage{float}
\usepackage{amsmath}
\usepackage{amssymb}


\usepackage[utf8]{inputenc}
\usepackage[a4paper, margin=1in]{geometry} % Controla los márgenes
\usepackage{titling}

\usepackage{mathbbol}

\title{Clase 10 }
\author{Manuel Garcia.}
\date{\today}

\renewcommand{\maketitlehooka}{%
  \centering
  \vspace*{0.05cm} % Espacio vertical antes del título
}

\renewcommand{\maketitlehookd}{%
  \vspace*{2cm} % Espacio vertical después de la fecha
}

\newcommand{\caja}[3]{%
  \begin{tcolorbox}[colback=#1!5!white,colframe=#1!25!black,title=#2]
    #3
  \end{tcolorbox}%
}

\begin{document}
\maketitle

\section{Campos Sobre Variedades Curvas }

\caja{black}{Apéndice A }{
  \textbf{Observador Acelerado de Rindler }
  De los principios de la teoria de la relatividad especial 
  \begin{itemize}
    \item Espacio de $ c $-vectores $ \mathcal M  $
      \begin{gather*}
        X \cdot X = - x ^ {0 } y ^ {0 } + x ^ {1 } y ^ {1 } + x ^ {2 }y ^ {2 } + x ^ {3 } y ^ {3 } =\eta _{\mu\nu} x ^ {\mu} x ^ {\nu} \qquad \text{para } X , Y \in \mathcal M   
      \end{gather*}
    \item Invariantes 
      \begin{gather*}
        \text{Con }\mathbb u, \mathbb a \in \mathcal M \quad \text{con } \mathbb u \text{ la c-velocidad y }\mathbb a \text{ es la c-aceleracion. }\\
        \mathbb u\cdot \mathbb u = \mathbb u ^ {2 } = - c ^ {2 } \\
        \mathbb a \cdot \mathbb u = 0 
      \end{gather*}
    \item 
      \begin{gather*}
        \mathbb u = \frac{d X(\tau ) }{d \tau } \qquad \qquad \mathbb u = \gamma(\vec u ) (c, \vec u ) \\
        \text{Con }\\
        \gamma(\vec u )= \frac{d t  }{d  \tau} = \frac{1}{\sqrt{1 - \frac{u ^2}{c ^2}} } \qquad \qquad \vec u = \frac{d \vec r (t) }{d t } \\
        \mathbb u ^ {2 } = \eta _{\alpha\beta} u ^ {\alpha} u ^ {\beta} = (u ^ {0 }, u ^ {1 }, u ^ {2 }, u ^ {3 }) = (c\gamma, \gamma u_x, \gamma u_y, \gamma u_z)
      \end{gather*}
      Se llega a 
      \begin{gather*}
        \mathbb a = \frac{d \mathbb u  }{d \tau} = \gamma(\vec u ) \left(\frac{d \gamma(\vec u ) }{d t} c , \quad \frac{d \gamma(\vec u ) }{d t } \vec u + \gamma(\vec u ) \vec a \right) 
      \end{gather*}
  \end{itemize}
}

\hfill 

\hfill 

\hfill 

\hfill 

\hfill 

\underline{Modelo canónico de una partícula desplazándose con aceleración propia ocnstante $ \vec a \equiv \vec g  $} esta es la misma aceleracion propia $ \mathbb a \cdot \mathbb a = \mathbb a ^ {2 } = (0,\vec g ) \cdot (0, \vec g ) = g ^ {2 } $ (en el sistema inercial de la particula un instante $ t = t_0  $).

Podemos reescribir como 
\begin{align*}
  \mathbb a ^ {2 } &= \eta _{\alpha\beta}  a ^ {\alpha\beta}; \qquad \qquad \mathbb a = (0,\vec g ) \\
  &= \eta _{00 }  (a ^ {0 } ) ^ {2 } + \eta _{11 }  a ^ {1 } a ^ {1 } \\
  &= - (a ^ {0 }) ^ {2 } + ( a ^ {1 }) ^ {2 } = \left|\vec g \right| ^ {2 }
\end{align*}
Tenemos que 
\begin{gather*}
  \mathbb a \cdot \mathbb u = - a ^ {0 } a ^ {0 } + a ^ {1 } a ^ {1 } = 0 \qquad \qquad u ^ {0 } u ^ {1 } = u ^ {1 } u ^ {0 } \quad \rightarrow \quad u ^ {1 } u ^ {0 } - u ^ {0 } u ^ {1 } = 0 \\
  - a ^ {0 } u ^ {0 } + a ^ {1 } u ^ {1 } = 0 \qquad \qquad - g u ^ {1} u ^ {0 } + g u ^ {0 } u ^ { 1 } = 0   
\end{gather*}
Entonces podemos plantear 
\begin{gather*}
   g u ^ {1 } = a ^ {0 } \quad \rightarrow \quad a ^ {0 } = \frac{d u ^ {0 } }{d \tau} = g u ^ {1 }\\
   g u ^ {0 } = a ^ {1 } \quad \rightarrow \quad a ^ {1 } = \frac{d u ^ {1 } }{d \tau} = g u ^ {0 }\\
\end{gather*}
Reescribiendo estos terminos 
\begin{gather*}
  \frac{d ^2 t  }{d \tau^2 } - g \frac{d x  }{d \tau} = 0 \qquad \qquad 
  \frac{d ^2 x  }{d \tau^2 } - g \frac{d t  }{d \tau} = 0 
\end{gather*}
Sean sus soluciones particulares 
\begin{gather*}
  t = g ^ {-1 } \sinh{g\tau} \\
  x = g ^ {-1 } \cosh{g\tau}
\end{gather*}
Esto es una hiperbola.
\begin{gather*}
  x^2 - t^2 = g ^ {-2 } 
\end{gather*}

\end{document}
