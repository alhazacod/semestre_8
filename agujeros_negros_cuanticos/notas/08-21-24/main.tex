\documentclass{article}

\usepackage[most]{tcolorbox}
\usepackage{physics}
\usepackage{graphicx}
\usepackage{float}
\usepackage{amsmath}
\usepackage{amssymb}


\usepackage[utf8]{inputenc}
\usepackage[a4paper, margin=1in]{geometry} % Controla los márgenes
\usepackage{titling}

\title{Clase Agujeros Negros Cuánticos }
\author{Manuel Garcia.}
\date{\today}

\renewcommand{\maketitlehooka}{%
  \centering
  \vspace*{0.05cm} % Espacio vertical antes del título
}

\renewcommand{\maketitlehookd}{%
  \vspace*{2cm} % Espacio vertical después de la fecha
}

\newcommand{\caja}[3]{%
  \begin{tcolorbox}[colback=#1!5!white,colframe=#1!25!black,title=#2]
    #3
  \end{tcolorbox}%
}

\begin{document}
\maketitle

\section{}
\begin{gather*}
  (\Box - m^2 ) \phi(x) = 0  
\end{gather*}
Métodos de Solución 
\begin{gather*}
  \phi^{(t)}_{\Omega} = \phi_\Omega(t,\vec x) \Theta_\epsilon 
\end{gather*}
Donde 
\begin{gather*}
  \phi_\Omega (t,\vec x ) = \phi_\Omega (r) Y _{lm } (\theta,\phi) \frac{1}{\sqrt{2 \left|\omega\right|} } e ^ {- i \omega t } 
\end{gather*}
Para $ r \rightarrow r_0 = 2M  $ 
\begin{gather*}
  \phi_\Omega ^ {\pm } (r) \approx  e ^ {\pm i \omega r_*} 
\end{gather*}
Con esto para $ +  $ tenemos 
\begin{gather*}
  \phi_\Omega \quad \rightarrow \quad \phi_\Omega ^ {+ } (t , \vec x) = Y _{lm }  (\theta, \phi) \frac{1}{\sqrt{2 \left|\omega\right|} } e ^ {- i \omega u }
\end{gather*}
Donde 
\begin{gather*}
  u \equiv t- r_* \\
  v \equiv t + r_* 
\end{gather*}
En general 
\begin{gather*}
  \phi_\Omega ^+ (t, \vec x) = \phi_\Omega ^ {out }(r) Y _{lm } (\theta ,\phi) \frac{1}{\sqrt{2 \left|\omega\right|} } e ^ {- i \omega u } 
\end{gather*}
La cual tiende al caso particular cuando $ r \rightarrow r_0  $.

Para -, tenemos 
\begin{gather*}
  \phi_\Omega^- (t, \vec x) = \phi_\Omega ^ {in }(r) Y _{lm } (\theta, \phi) \frac{1}{\sqrt{2 \left|\omega\right|} } e ^ {- i \omega v } 
\end{gather*}
Podemos reescribirlas como 
\begin{gather*}
  \phi_\Omega ^ {+ } (t,\vec x) = \phi_\Omega ^ {out } (u, \vec x) \qquad \qquad \phi_\Omega ^ {- } (t,\vec x) = \phi_\Omega ^ {in } (v, \vec x) 
\end{gather*}
Haciendo la transformación para 
\begin{gather*}
   U = T-Z \\
   V = T+Z
\end{gather*}
Donde $ T,Z  $ son las coordenadas de Kruskal 
\begin{gather*}
  \phi_\Omega ^ {(\epsilon) } (U, \vec x ) = \Theta (- \epsilon U ) \phi_\Omega (u,\vec x)  \qquad \qquad 
  \phi_\Omega ^ {(\epsilon) } (V, \vec x ) = \Theta (\epsilon V ) \phi_\Omega (u,\vec x)  
\end{gather*}

El conjunto $ \{\phi_\Omega ^ {(\epsilon)}\} $ es un conjunto de modos que satisfacen las relaciones de ortogonalidad con los productos escalares siguientes 
\begin{gather*}
  (\phi_\Omega ^ {(\epsilon)}, \phi_{\Omega'} ^ {(t')}) = \epsilon\ \epsilon (\omega) \delta _{\Omega \Omega' } \delta _{\epsilon\epsilon'}  
\end{gather*}
Donde 
\begin{gather*}
  \epsilon(\omega) = sign(\omega)  \\
  \delta _{\Omega \Omega'} = \delta(\omega-\omega' ) \delta _{ll'} \delta _{mm'} \delta _{kk'}   \qquad \qquad ,\ k = \pm 1  
\end{gather*}
Ademas $ \{\phi_\Omega ^ {(\epsilon)}\} $ forma un conjunto completo \\
Con respecto a la relación de ortogonalidad 
\begin{gather*}
  \displaystyle\int_{}^{}d^3 \vec x \ \ \gamma( \vec x) \phi _{\Omega' } ^* (\vec x ) = \delta _{\Omega \Omega '}  
\end{gather*}

\end{document}
