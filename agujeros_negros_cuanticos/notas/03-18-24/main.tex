\documentclass{article}

\usepackage[most]{tcolorbox}
\usepackage{physics}
\usepackage{graphicx}
\usepackage{float}
\usepackage{amsmath}
\usepackage{amssymb}


\usepackage[utf8]{inputenc}
\usepackage[a4paper, margin=1in]{geometry} % Controla los márgenes
\usepackage{titling}

\usepackage{mathbbol}

\usepackage{tikz}
\usepackage{pgfplots}
\pgfplotsset{compat=1.18}

\title{Clase 11}
\author{Manuel Garcia.}
\date{\today}

\renewcommand{\maketitlehooka}{%
  \centering
  \vspace*{0.05cm} % Espacio vertical antes del título
}

\renewcommand{\maketitlehookd}{%
  \vspace*{2cm} % Espacio vertical después de la fecha
}

\newcommand{\caja}[3]{%
  \begin{tcolorbox}[colback=#1!5!white,colframe=#1!25!black,title=#2]
    #3
  \end{tcolorbox}%
}

\begin{document}
\maketitle

\section{Campos Sobre Variedades Curvas }

\caja{black}{Apéndice A }{
  \textbf{Observador Acelerado de Rindler }
  De los principios de la teoria de la relatividad especial 
  \begin{itemize}
    \item Espacio de $ c $-vectores $ \mathcal M  $
      \begin{gather*}
        X \cdot X = - x ^ {0 } y ^ {0 } + x ^ {1 } y ^ {1 } + x ^ {2 }y ^ {2 } + x ^ {3 } y ^ {3 } =\eta _{\mu\nu} x ^ {\mu} x ^ {\nu} \qquad \text{para } X , Y \in \mathcal M   
      \end{gather*}
    \item Invariantes 
      \begin{gather*}
        \text{Con }\mathbb u, \mathbb a \in \mathcal M \quad \text{con } \mathbb u \text{ la c-velocidad y }\mathbb a \text{ es la c-aceleracion. }\\
        \mathbb u\cdot \mathbb u = \mathbb u ^ {2 } = - c ^ {2 } \\
        \mathbb a \cdot \mathbb u = 0 
      \end{gather*}
    \item 
      \begin{gather*}
        \mathbb u = \frac{d X(\tau ) }{d \tau } \qquad \qquad \mathbb u = \gamma(\vec u ) (c, \vec u ) \\
        \text{Con }\\
        \gamma(\vec u )= \frac{d t  }{d  \tau} = \frac{1}{\sqrt{1 - \frac{u ^2}{c ^2}} } \qquad \qquad \vec u = \frac{d \vec r (t) }{d t } \\
        \mathbb u ^ {2 } = \eta _{\alpha\beta} u ^ {\alpha} u ^ {\beta} = (u ^ {0 }, u ^ {1 }, u ^ {2 }, u ^ {3 }) = (c\gamma, \gamma u_x, \gamma u_y, \gamma u_z)
      \end{gather*}
      Se llega a 
      \begin{gather*}
        \mathbb a = \frac{d \mathbb u  }{d \tau} = \gamma(\vec u ) \left(\frac{d \gamma(\vec u ) }{d t} c , \quad \frac{d \gamma(\vec u ) }{d t } \vec u + \gamma(\vec u ) \vec a \right) 
      \end{gather*}
  \end{itemize}
}

\textbf{Aceleración Propia: } 
\begin{gather*}
  \vec u = 0 \qquad \qquad \text{Instantanea }\\
  \mathbb{a} = (0, \vec a)
\end{gather*}
\begin{gather*}
  \mathbb a ^ {2 } = \eta _{\alpha\beta} a ^ {\alpha} a ^ {\beta} = \left|\vec a \right| ^ {2 } \qquad \qquad \mathbb a \in \mathcal M  
\end{gather*}

\textbf{Modelo: Observador de Ridler} $ \vec a = \vec g = cnst.  $

\textbf{Trayectoria } 
\begin{gather*}
  \frac{d ^2 t  }{d \tau^2 } - g \frac{d x }{d \tau }= 0 \\
  \frac{d ^2 x  }{d \tau } - g \frac{d  t  }{d \tau } = 0 \\
  \text{Con soluciones particulares }\\
  t = g ^ {-1 }\sinh{g\tau }\qquad \qquad x = g ^ {-1 } \cosh{g\tau}\\
  x^2 - t^2 = g ^ {-2 }
\end{gather*}
Transformaciones 
\begin{gather*}
  t(\tau) = g ^ {-1 } \sinh{g\tau}\\ 
  x(\tau) = g ^ {-1 } \cosh{g\tau} \\
  y = y \\
  z = z 
\end{gather*}

\begin{tikzpicture}
\begin{axis}[
legend pos=outer north east,
axis lines = box,
xlabel = $t$,
ylabel = $x$,
variable = t,
trig format plots = rad,
]
\addplot [
  domain=-4:4,
  samples=200,
  color=blue,
  ]
  {-(1+x^2)^(1/2)};
\addlegendentry{$x^2 - t^2 = 1 $}

\addplot [
  domain=-4:4,
  samples=200,
  color=blue,
  ]
  {(1+x^2)^(1/2)};

\end{axis}

\end{tikzpicture}


\caja{green}{}{
  \begin{gather*}
    t = a ^ {-1 } e ^ {a ^ {3 }} \sinh{a\eta}\\
    x = a ^ {-1 } e ^ {a ^ {\xi }} \cosh{a\eta}
  \end{gather*}
  Donde 
  \begin{gather*}
    g:=  e ^ {-a \xi }\\
    \tau:=  e ^ {a \xi }
  \end{gather*}
}

\textbf{Metrica }
\begin{gather*}
  dS ^ {2 } = - dt ^ {2 } + dx ^ {2 } + dy ^ {2 } + dz ^ {2 } \\
  \rightarrow dS ^ {2 } = - e ^ {2a \xi }d\eta ^ {2 } + e ^ {2a \xi } d \xi^2 + dy^2 + dz^2 
\end{gather*}

De las tranformaciones: 
\begin{gather*}
  \text{Si }\eta = cte \qquad \forall \quad \xi \\
  t = a ^ {-1 } e ^ {a \xi } \sinh cte \\
  x = a ^ {-1 } e ^ {a \xi } \cosh cte \\
  \frac{t}{x} = \tanh{cte} = cte
\end{gather*}

\begin{tikzpicture}
\begin{axis}[
legend pos=outer north east,
axis lines = box,
xlabel = $t$,
ylabel = $x$,
variable = t,
trig format plots = rad,
]
\addplot [
  domain=-4:4,
  samples=200,
  color=blue,
  ]
  {(1+x^2)^(1/2)};
\addlegendentry{$\xi = cte$}

\addplot [
  domain=0:2,
  samples=200,
  color=red,
  ]
  {2*x};
\addlegendentry{$\eta = cte$}

\addplot [
  domain=-2:4,
  samples=200,
  color=black,
  ]
  {x};

\addplot [
  domain=-4:2,
  samples=200,
  color=black,
  ]
  {-x};

\end{axis}

\end{tikzpicture}

Esto se puede ver como una "rejila" formada por el eje $ \eta  $ donde $ \xi = cte  $ y el eje $ \xi  $ donde $ \eta = cte  $. Aparece un nuevo tiempo $ \eta  $, el tiempo Miskowskiano.



\end{document}
