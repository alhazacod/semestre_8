\documentclass{article}

\usepackage[most]{tcolorbox}
\usepackage{physics}
\usepackage{graphicx}
\usepackage{float}
\usepackage{amsmath}
\usepackage{amssymb}


\usepackage[utf8]{inputenc}
\usepackage[a4paper, margin=1in]{geometry} % Controla los márgenes
\usepackage{titling}

\title{Clase 5}
\author{Manuel Garcia.}
\date{\today}

\renewcommand{\maketitlehooka}{%
  \centering
  \vspace*{0.05cm} % Espacio vertical antes del título
}

\renewcommand{\maketitlehookd}{%
  \vspace*{2cm} % Espacio vertical después de la fecha
}

\newcommand{\caja}[3]{%
  \begin{tcolorbox}[colback=#1!5!white,colframe=#1!25!black,title=#2]
    #3
  \end{tcolorbox}%
}

\begin{document}
\maketitle

\section{Teoria de perturbaciones independiente del tiempo}
Tenemos $ \{\phi_n (x), E_n ^ {(0 )}\} $
\begin{gather*}
  H_0 \phi_n (x) = E_n ^ {(0)} \phi_n (x)  
\end{gather*}
Vamos a hacer 
\begin{gather*}
  H = H_0 +\lambda \omega \\
  E_n = E_n ^ {(0)} + \lambda E_n ^ {(1)} + \lambda^2 E_n ^ {(2) } + \cdots = \displaystyle\sum_{k = 0 }^{ \infty} \lambda ^ {k } E_n ^ {(k)}\\
  \phi_n = \phi_n ^ {(0)} + \lambda \phi_n ^ {(1)} + \lambda^2 \phi_n ^ {(2)} + \cdots = \displaystyle\sum_{k = 0 }^{\infty} \lambda^k \phi_n ^ {(k)}
\end{gather*}
Entonces podemos reescribir 
\begin{gather*}
  (H_0 + \lambda \omega) \displaystyle\sum_{k = 0 }^{\infty} \lambda ^ {k } \phi_n ^ {(k)} = \displaystyle\sum_{l = 0 }^{\infty} \lambda ^ {l } E_n ^ {(l)} \displaystyle\sum_{r = 0 }^{\infty} \lambda ^ {r } \phi_n ^ {(r)} = \displaystyle\sum_{l = 0 }^{} \displaystyle\sum_{ r = 0 }^{} \lambda ^ {l + r } E _{n } ^ {l } \phi_n ^ {(r)}
\end{gather*}
Hacemos $ l + r = 1  $ 
\begin{gather*}
  H_0 \phi_n + \lambda \omega \phi_n ^ {(0 )} + \cdots = E_n ^ {(0)}\phi_n + \lambda(E_n ^ {(0) }\phi_n ^ {(1)} + E_n ^ {(1)}\phi_n ) 
\end{gather*}


\end{document}
