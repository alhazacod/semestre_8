\documentclass{article}

\usepackage[most]{tcolorbox}
\usepackage{physics}
\usepackage{graphicx}
\usepackage{float}
\usepackage{amsmath}
\usepackage{amssymb}


\usepackage[utf8]{inputenc}
\usepackage[a4paper, margin=1in]{geometry} % Controla los márgenes
\usepackage{titling}

\title{Clase 1 }
\author{Manuel Garcia.}
\date{\today}

\renewcommand{\maketitlehooka}{%
  \centering
  \vspace*{0.05cm} % Espacio vertical antes del título
}

\renewcommand{\maketitlehookd}{%
  \vspace*{2cm} % Espacio vertical después de la fecha
}

\newcommand{\caja}[3]{%
  \begin{tcolorbox}[colback=#1!5!white,colframe=#1!25!black,title=#2]
    #3
  \end{tcolorbox}%
}

\begin{document}
\maketitle

\section{Momento Angular }
\begin{gather*}
  \vec L = \vec r \cross \vec p \\
  L_k  = \epsilon _{ijk } \hat r _i \hat p _j = + i \hbar \epsilon _{ijk }  \hat r _i \partial _j = - i \hbar \epsilon _{ijk } x_i \frac{\partial  }{\partial x ^ {j }}\\
  [L] = \hbar = d \cross m \cross v = m \frac{d ^2}{t ^2} t = F d t = E t 
\end{gather*}
 
\subsection{Algebra de Lee }
\begin{gather*}
  \{L_x,L_y,L_z\} 
\end{gather*}
\begin{gather*}
  [, ] : V \otimes V \qquad \rightarrow \qquad V \\
  (L_i , L_j ) \qquad \rightarrow \qquad  [L_i, L_j ] = i \hbar \epsilon _{ijk } L_k 
\end{gather*}

\textbf{Propiedades }
\begin{itemize}
  \item $ [\hat x_i, \hat p_j ] = i \hbar \delta _{ij } \quad \rightarrow \quad \Delta x \Delta p_x  \geq \frac{\hbar }{2} $
  \item $ [AB,C ] = A[B,C] + [A,C]B  $
  \item $ [A,BC] = B[A,C ] +  [A,B]C  $
\end{itemize}

\begin{gather*}
  [L_i, L_j ] = [i \hbar \epsilon _{ipq } x_p p_q, i \hbar \epsilon _{imn } x_m p _n ] \\
  = - \hbar ^2 \epsilon _{ipq } \epsilon _{jmn } [\hat x _p \hat p _q , \hat x _m \hat p _n ]
\end{gather*}
Usando las propiedade spodemos escribir lo que está entre parentesis como: 
\begin{gather*}
x\{x_p [p_q, x_m ] p_n + x_px_m \underset{=0}{[p_q,p_n ]} + \underset{ = 0 }{[x_p, x_n ]} p_n p_q + x_m\underset{= i \hbar \delta _{pn }}{[x_p,p_n]}p_q\}
\end{gather*}
Entonces 
\begin{gather*}
  [L_i,L_j] = \epsilon _{ipq }  \epsilon _{jmn } \{ - i \hbar  x_p p_n \delta _{qm }  + i \hbar x_m p_q \delta _{pn }  \} \\
  = i \hbar \epsilon _{inq }  \epsilon _{jmn } x _{m }  p _{q } - i \hbar  \epsilon _{ipm }  \epsilon _{jmn } x _{p } p _{n } \\
  \text{Reescribiendo los indices mudos }\\
  = i \hbar  \hat x_p \hat p _n \underset{ = \delta _{ij } \delta _{pn } - \delta _{in } \delta _{pj } - \delta _{ji }  \delta _{pn }  + \delta _{jn } \delta _{pi }  }{\{\epsilon _{ipm }  \epsilon _{jnm } - \epsilon _{jpm } \epsilon _{inm}  \}} \\
  = i \hbar \{- \hat x _j p_i + x_ip_j \} \\
  = i \hbar (x_i p_j - x_j p_i ) \\
  = i \hbar \epsilon _{ijk } L_k
\end{gather*}

\begin{itemize}
  \item $ [L_x,L_y] = i \hbar L_z $
  \item $ [L_x,L_z] = i \hbar L_y $
  \item $ [L_y,L_z] = i \hbar L_x $
\end{itemize}

\hfill 

\hfill 

\begin{gather*}
  [L^2, L_j ] = [L_iL_i, L_j ] = L_i[L_i,L_j] + [L_i,L_j]L_i\\
  = i \hbar \epsilon _{ijk } \hat L_i \hat L_k + i \hbar \epsilon _{ijk } L_k L_i\\
  = i \hbar \epsilon _{ijk } \{L_i L_k + L_k L_i \} = 0 
\end{gather*}

Podemos decir que $ L^2  $ conmuta con todas las bases del espacio vectorial.

\hfill

Recordemos que $ \{L_x,L_y,L_z \} $ son operadores hermiticos

\hfill

Vamos a definir el operador $ L _{\pm } = L_x \pm i L_y  $ (este NO es hermitico)
\begin{gather*}
  \frac{1}{2} (L_+ L_- + L_- L_+ ) = L_x ^2 + L_y ^2 
\end{gather*}
Entonces podemos escribir $ L^2  $ como: 
\begin{gather*}
  L^2 = L_x^2 + L_y^2 + L_z^2 = \frac{1}{2} (L_+L_- + L_-L_+) 
\end{gather*}

\begin{itemize}
  \item $ [L_+, L_- ] = 2 \hbar L_z  $
  \item $ [L_z, L_+ ] = \hbar L_+  $ 
  \item $ [L_z, L_- ] = - \hbar L_- $
\end{itemize}

Tenemos que $ L^2  $ conmuta con todos, entonces: 
\begin{gather*}
  [L^2, L_\pm ] = 0 
\end{gather*}

Tenemos que $ \{L^2, L_z \} $ conmutan entre sí 

\begin{gather*}
  \{H, L^2 , L_z \} \qquad \qquad \text{donde }H = \frac{p^2 }{2m } + \frac{1}{r}
\end{gather*}

El hamiltoniano es central, no depende de $ \theta  $ ni de $ \phi  $y como $ L^2  $ y $ L_z  $ son derivadas de estas variables entonces:
\begin{gather*}
  [H, L^2 ] = [H,L_z ] = 0  
\end{gather*}

$ {L^2,L_z } $ \textbf{son constantes de movimiento.}

\hfill

Como $ \{H,L^2,L_z \} $ conmutan entre sí entonces tenemos una autofuncion $ \Phi _{nlm }  $, como $ L^2,L_z  $ son constantes de movimiento entonces $ l,m  $ no van a cambiar, a estos se les conoce como numeros cuanticos.

Vamos a hacer $ \{L^2, L_z \} \rightarrow \ket{\alpha, \beta} $
\begin{itemize}
  \item $ L^2 \ket{\alpha, \beta} = \hbar ^2 \alpha \ket{\alpha, \beta} $ 
  \item $ L_z \ket{\alpha, \beta} = \hbar \beta \ket{\alpha, \beta} $
\end{itemize}

Recordemos que $ [L^2, L _{@pm } ] =0 $ 

\begin{gather*}
  L^2 (L_\pm  \ket{\alpha,\beta}) = L_\pm  (L^2 \ket{\alpha,\beta}) = L_\pm (\hbar \alpha \ket{\alpha,\beta}) = \hbar \alpha (L_\pm \ket{\alpha,\beta}) 
\end{gather*}

\hfill 

\begin{gather*}
  L_z (L_\pm \ket{\alpha,\beta}) = (L_\pm L_z \pm \hbar L_\pm ) \ket{\alpha,\beta} \\
  = (L_\pm \hbar \beta \pm \hbar L_\pm  ) \ket{\alpha, \beta}\\
  = \hbar (\beta \pm 1 )(L_\pm \ket{\alpha, \beta})
\end{gather*}

Y tenemos que $ L_+ \ket{\alpha, \beta} \rightarrow \hbar (\beta + 1 ) $ y $ L_- \ket{\alpha,\beta} \rightarrow \hbar (\beta-1 ) $

Vamos a realizar la siguiente aproximacion: 
\begin{gather*}
  L_\pm \ket{\alpha,\beta} \approx C _{i,p } \ket{\alpha,\beta\pm 1} 
\end{gather*}

Podemos hacer lo siguiente: 
\begin{gather*}
  \bra{\alpha,\beta} L_\mp L_\pm \ket{\alpha,\beta} = (L_\pm \ket{\alpha,\beta})^+ (L_\pm \ket{\alpha,\beta}) = \left|L_\pm \ket{\alpha,\beta}\right|^2 
\end{gather*}


\begin{gather*}
  \bra{\alpha,\beta} L _{\pm }  L _{\pm } \ket{\alpha, \beta} =  
\end{gather*}

\end{document}
